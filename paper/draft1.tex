% ============================================================
%  AMORTISATION WRITE-UP  —  SECTION 1
% ============================================================

\documentclass[12pt,letterpaper]{article}
\usepackage[margin=1in]{geometry}
\usepackage{amsmath,amssymb,amsthm}
\usepackage{enumitem}
\usepackage{tcolorbox}
\usepackage{fancyhdr}
\usepackage{mathrsfs}
\usepackage[hidelinks]{hyperref}
\usepackage{booktabs}
\usepackage{tabularx}
\usepackage{multirow}
\usepackage[english]{babel}
\usepackage[autostyle, english = american]{csquotes}
\MakeOuterQuote{"}

% Custom theorem environments
\theoremstyle{definition}
\newtheorem{definition}{Definition}[section]
\newtheorem{example}{Example}[section]
\newtheorem{exercise}{Exercise}[section]

\theoremstyle{plain}
\newtheorem{theorem}{Theorem}[section]
\newtheorem{lemma}[theorem]{Lemma}
\newtheorem{proposition}[theorem]{Proposition}
\newtheorem{corollary}[theorem]{Corollary}

\theoremstyle{remark}
\newtheorem{remark}{Remark}[section]
\newtheorem*{note}{Note}

\theoremstyle{definition}

\title{\textbf{Simulation and Analysis of Sector-Apportioned Intensity Models for Correlated Defaults:\\Integrating repayment schedules}}
\author{Siddharth Kamlesh Sharma, Devang Sinha, Shashi Jain, Srikanth K. Iyer}
\date{}

\begin{document}

\maketitle

% ============================================================
%  AMORTISATION WRITE-UP  —  SECTION 1
%  Style follows: Monte_Carlo_Paper_Draft_Dec_2025
%  Parameters and schedules faithful to stochastic_lgd.ipynb
% ============================================================

\section{Introduction, Motivation, and Model Setup}

\subsection{Introduction}

The sector-apportioned intensity framework developed in \cite{sinha2025} provides a
tractable and computationally efficient approach for analysing correlated defaults
arising from sectoral shocks. In that framework each obligor's default intensity is
decomposed as
\begin{equation}
  \lambda^A_t \;=\; X^A \;+\; \sum_{j=1}^{J} w^{Aj}\,Y^j_t,
  \label{eq:base_intensity}
\end{equation}
where $X^A$ is a constant idiosyncratic component, $Y^j_t$ denotes the stochastic
intensity of sector $j$ governed by a CIR-type diffusion with loss-driven feedback,
and $w^{Aj} \geq 0$ measures obligor $A$'s exposure to that sector. Portfolio losses
enter through a feedback term $\delta^{Aj}\,\mathrm{d}L_t$ in the dynamics of each
$Y^j_t$, generating the self-exciting default clustering that is characteristic of
credit crises.

A central modelling assumption in \cite{sinha2025}, shared by much of the intensity-based
credit-risk literature, is that the loss incurred upon the $n$-th default is an
independent draw $\ell_n \sim \mathcal{U}[0,1]$. Equivalently, the
\emph{exposure at default} (EAD) of every obligor is treated as constant through
time and the loss given default (LGD) is implicitly set to unity. Whilst this
simplification does not distort the sector-concentration and contagion results that
are the primary focus of that paper, it abstracts away a crucial dimension of
real-world credit risk: the outstanding principal of a loan evolves according to its
contractual repayment schedule, and the profile of this evolution differs markedly
across the loan structures that populate bank portfolios.

The present paper extends the framework of \cite{sinha2025} along two complementary
dimensions. First, we replace the static, uniform loss draw with a
\emph{time-varying} EAD schedule $E^A(t)$, so that the loss incurred when obligor
$A$ defaults at time $\tau^A$ is
\begin{equation}
  \ell^A \;=\; \mathrm{LGD}^A \cdot E^A(\tau^A),
  \label{eq:loss_general}
\end{equation}
where $E^A(\tau^A)$ is the outstanding principal at the moment of default under the
obligor's contracted repayment schedule. Second, we model LGD as stochastic,
drawing $\mathrm{LGD}^A \sim \mathrm{Beta}(\alpha, \beta)$ independently for each
default event, replacing the implicit assumption of full loss on exposure. The
systematic intensity dynamics and contagion mechanism of \eqref{eq:base_intensity}
are inherited unchanged; the contribution of this work lies entirely in enriching the
loss side of the model. We study four canonical amortisation contracts --- Bullet,
Linear (Italian), French, and Negative Amortisation --- and demonstrate that the
choice of repayment structure has material consequences for portfolio tail risk,
expected shortfall, and systemic resilience.

\subsection{Motivation and Literature Review}

\paragraph{Why amortisation matters for credit risk.}
The importance of time-varying exposure in credit portfolios has long been recognised
in the regulatory capital literature. The Basel~II and Basel~III frameworks permit
banks to use internal estimates of EAD for revolving facilities, acknowledging that
the drawn amount at default need not equal the committed line \cite{bis2006}. For
term loans and mortgages the outstanding balance follows a deterministic schedule
conditional on no prepayment, yet the \emph{timing} of default relative to that
schedule introduces material variability in realised losses. A borrower who defaults
early in a bullet loan's life imposes a near-full-principal loss; the same event
late in a well-amortised loan leaves far less outstanding. This simple observation
implies that two portfolios with identical notional values and identical intensity
processes can have substantially different loss distributions if their repayment
architectures differ.

Early portfolio credit-risk models --- the Vasicek one-factor model \cite{vasicek2002},
CreditMetrics \cite{gupton1997}, and CreditRisk$+$ \cite{csfp1997} --- were largely
silent on this heterogeneity, treating EAD as a deterministic scalar. The first
systematic treatment of maturity and amortisation within a reduced-form framework
appears in \cite{duffie1999}, who embed time-varying cashflows into affine survival
probabilities. Subsequent work formalises the notion that the credit exposure profile
--- the expected EAD conditional on survival to time $t$ --- is a fundamental input
to economic capital calculations, particularly for instruments with long maturities
and structured repayment schedules \cite{lando2004, mcneil2015}.

\paragraph{Amortisation and systemic risk.}
The interaction between amortisation schedules and \emph{systemic} risk is less
well-studied, yet highly policy-relevant. During the 2007--2009 financial crisis, a
disproportionate share of losses arose from instruments whose EAD was either constant
to maturity or actively growing prior to a contractual reset \cite{gorton2010,
brunnermeier2009}. By contrast, portfolios of traditional instalment loans amortised
steadily and presented declining exposures even as the underlying sectors came under
stress. This empirical contrast motivates a structural comparison of repayment
architectures within a contagion-sensitive intensity model.

Within the intensity-based literature, \cite{errais2010} and \cite{aitsahalia2015}
model default clustering via self-exciting processes but maintain static notional
exposures. \cite{deshpande2009} extend the CreditRisk$+$ sector framework to
correlated sector drivers but likewise hold EAD fixed. The present work fills this
gap by coupling the sector-apportioned intensity of \cite{sinha2025} with
contractually specified, time-varying EAD schedules, thereby enabling a clean
decomposition of how repayment structure interacts with contagion strength to shape
the portfolio loss distribution.

\paragraph{Stochastic recovery.}
A further departure from the base model concerns the recovery rate. Empirical evidence
consistently documents that recovery rates are stochastic and, crucially, negatively
correlated with default rates: recoveries fall precisely when defaults cluster,
amplifying systemic losses \cite{altman2005, frye2000}. To capture this effect in a
tractable manner we model $\mathrm{LGD}^A \sim \mathrm{Beta}(\alpha, \beta)$, drawn
independently across obligors and simulation paths. The Beta distribution is the
canonical choice in both regulatory and academic practice \cite{schuermann2004,
bis2006}: it is supported on $[0,1]$, accommodates the full range of empirical
skewness, and its parameters can be moment-matched directly from historical workout
data. In accordance with standard regulatory assumptions we calibrate to a mean LGD
of $45\%$ and a standard deviation of $15\%$, yielding $\alpha \approx 4.5$ and
$\beta \approx 5.5$ via the moment-matching relations
\begin{equation}
  \nu \;=\; \frac{\mu(1-\mu)}{\sigma^2} - 1, \qquad
  \alpha \;=\; \mu\,\nu, \qquad
  \beta \;=\; (1-\mu)\,\nu,
  \label{eq:beta_params}
\end{equation}
with $\mu = 0.45$ and $\sigma = 0.15$.

\subsection{Model Setup}
\label{sec:setup}

\subsubsection{Amortisation Schedules}

We consider a portfolio of $N$ obligors. Each obligor $A$ holds a loan of principal
$P^A$, originated at $t = 0$ with contractual maturity $T$. The outstanding balance
(EAD) at time $t$ is denoted $E^A(t)$ and is determined entirely by the repayment
contract. Schedules are implemented in discrete time with $N_p$ payment periods per
year, so that the period length is $\Delta = 1/N_p$, the total number of periods is
$n_{\mathrm{tot}} = \lceil T N_p \rceil$, and the number of payments made by time
$t$ is $n(t) = \min\!\left(\lfloor t N_p \rfloor,\, n_{\mathrm{tot}}\right)$.
We study four canonical structures, each defined below for a loan with principal
$P^A$ and contractual rate $r$.

\paragraph{(i) Bullet loan.}
No principal is repaid prior to maturity; the full notional is outstanding until $T$:
\begin{equation}
  E_{\mathrm{B}}(t) \;=\; \begin{cases} P^A, & t < T,\\ 0, & t \geq T. \end{cases}
  \label{eq:bullet}
\end{equation}
The EAD profile is constant and the lender bears full principal exposure for the
entire life of the loan.

\paragraph{(ii) Linear (Italian) amortisation.}
Principal is repaid in equal instalments of size $P^A / n_{\mathrm{tot}}$ per period.
The outstanding balance after $n(t)$ payments is
\begin{equation}
  E_{\mathrm{L}}(t) \;=\; P^A\!\left(1 - \frac{n(t)}{n_{\mathrm{tot}}}\right).
  \label{eq:linear}
\end{equation}
Exposure declines piecewise linearly, so that by mid-life roughly half the principal
has been returned to the lender.

\paragraph{(iii) French (constant-annuity) amortisation.}
Total debt-service is constant across all periods. With per-period rate
$r_\Delta = r / N_p$, the constant annuity payment is
\begin{equation}
  C \;=\; P^A\,\frac{r_\Delta}{1 - (1+r_\Delta)^{-n_{\mathrm{tot}}}},
  \label{eq:annuity_payment}
\end{equation}
and after $n(t)$ payments have been made, $n_r(t) = n_{\mathrm{tot}} - n(t)$ periods
remain. The outstanding balance equals the present value of the remaining annuity
stream:
\begin{equation}
  E_{\mathrm{F}}(t) \;=\; C\,\frac{1 - (1+r_\Delta)^{-n_r(t)}}{r_\Delta}.
  \label{eq:french}
\end{equation}
Early payments are predominantly interest, so principal declines slowly at first and
accelerates towards maturity. This schedule governs the vast majority of residential
mortgages and consumer instalment loans.

\paragraph{(iv) Negative amortisation.}
Scheduled payments are insufficient to cover accruing interest; the shortfall is
capitalised each period, so the outstanding balance grows monotonically throughout
the loan's life:
\begin{equation}
  E_{\mathrm{N}}(t) \;=\; \begin{cases}
    P^A\,(1 + r_\Delta)^{n(t)}, & t < T,\\
    0, & t \geq T.
  \end{cases}
  \label{eq:negam}
\end{equation}
Any default occurring during $[0, T)$ is therefore associated with an exposure
strictly exceeding the origination principal, making this the most credit-risk-adverse
of the four structures. This contract appeared prominently in option-ARM mortgage
products of the mid-2000s.

The four EAD profiles are illustrated in Figure~\ref{fig:ead_profiles}. The contrast
between the declining profiles of the Linear and French schedules and the flat or
growing profiles of the Bullet and Negative Amortisation contracts is the central
object of study in this paper.

\begin{figure}[ht]
  \centering
  \includegraphics[width=0.75\textwidth]{figures/ead_profiles.png}
  \caption{EAD profiles $E(t)$ over the loan horizon $[0, T]$ for the four
  amortisation contracts: Bullet (constant at par), Linear/Italian (piecewise
  linear decline), French (convex decline under constant-annuity payments), and
  Negative Amortisation (monotonically growing via interest capitalisation).
  Parameters: unit principal $P = 10$, maturity $T = 5$, annual rate $r = 0.12$,
  $N_p = 12$ payment periods per year.}
  \label{fig:ead_profiles}
\end{figure}

\subsubsection{Cumulative Exposure Mass}
\label{sec:cumulative_mass}

Whilst the EAD profiles in Figure~\ref{fig:ead_profiles} characterise the
\emph{instantaneous} outstanding balance under each contract, a complementary
perspective is provided by the \emph{cumulative exposure mass}, defined as the
time-integrated outstanding balance normalised by the product of principal and
maturity:
\begin{equation}
  \mathcal{M}(t) \;=\; \frac{1}{P\,T}\int_0^t E(s)\,\mathrm{d}s.
  \label{eq:cumulative_mass}
\end{equation}
The quantity $\mathcal{M}(t)$ measures what fraction of the maximum possible
credit exposure --- a unit notional held for the full term --- has been borne
by the lender up to time $t$. Its derivative $\mathcal{M}'(t) = E(t)/(PT)$
is proportional to the instantaneous EAD, so the curvature of
$\mathcal{M}(\cdot)$ directly reflects how rapidly outstanding principal is
being returned.

Figure~\ref{fig:cumulative_mass} plots $\mathcal{M}(t)$ for the four contracts
over the five-year horizon used in the experiments. Several features are
noteworthy. The Bullet contract accumulates mass linearly, reflecting its
constant exposure; its curve is a straight diagonal and serves as an upper
reference. The Negative Amortisation contract exceeds even this benchmark for
all $t \in (0,T)$, since capitalised interest causes the outstanding balance
to grow above par throughout the loan's life. Both findings are consistent
with the expectation that late- or non-amortising structures impose the
greatest credit exposure on the lender.

The Linear (Italian) and French curves lie below the Bullet, but differ
subtly in their degree of concavity. The Linear schedule declines at a
constant rate from origination, so $\mathcal{M}_L(t) = t/T - t^2/(2T^2)$
and the associated curvature $\mathcal{M}_L''(t) = -1/T^2$ is uniform
throughout the horizon. The French schedule, by contrast, exhibits a convex
EAD profile: early instalments are predominantly interest, so $E_F(t)$
remains close to par in the first portion of the loan's life and only begins
to fall steeply as the principal component of each payment grows. As a
consequence, $\mathcal{M}_F(t)$ tracks the Bullet curve closely in early
periods and curves away from it only in the later half of the horizon,
concentrating its concavity towards maturity. Integrated over the full term,
the Linear contract therefore displays slightly more pronounced and uniformly
distributed curvature than the French, even though the French contract
ultimately bears a greater total exposure mass $\mathcal{M}_F(T) >
\mathcal{M}_L(T)$. This distinction has practical implications for loss
modelling: under the sector-apportioned intensity framework, defaults that
arrive in the early-to-middle portion of the horizon --- which are the most
likely under typical CIR-level intensities --- interact with an EAD that
is, on average, higher for French than for Linear obligors, a difference
that will be reflected in the tail metrics reported in Section~\ref{sec:tail}.

\begin{figure}[ht]
  \centering
  \includegraphics[width=0.75\textwidth]{figures/cum_exm_t.png}
  \caption{Normalised cumulative exposure mass $\mathcal{M}(t)$ for the four
  amortisation contracts over a five-year horizon ($T = 5$, $r = 0.12$,
  $N_p = 12$). The Negative Amortisation curve lies above the Bullet
  reference due to interest capitalisation. The Linear contract exhibits
  slightly more pronounced and uniformly distributed concavity than the
  French contract, whose curvature is concentrated towards maturity owing
  to the convex shape of the constant-annuity EAD profile.}
  \label{fig:cumulative_mass}
\end{figure}

\subsubsection{Stochastic Loss Given Default}

The loss incurred when obligor $A$ defaults at time $\tau^A$ is
\begin{equation}
  \ell^A \;=\; \mathrm{LGD}^A \cdot E^A(\tau^A),
  \label{eq:lgd_full}
\end{equation}
where $\mathrm{LGD}^A \sim \mathrm{Beta}(\alpha, \beta)$ is drawn independently for
each default event. With $(\alpha, \beta) \approx (4.5, 5.5)$, the distribution has
mean $0.45$ and standard deviation $0.15$, consistent with regulatory benchmarks for
senior obligations. The cumulative portfolio loss up to time $t$ retains the
structure of \cite{sinha2025}:
\begin{equation}
  L_t \;=\; \sum_{n \geq 1} \ell^{A_n}\,\mathbf{1}_{\{T_n \leq t\}},
  \label{eq:loss_process}
\end{equation}
where $T_n$ is the $n$-th default time and $A_n$ is the identity of the defaulting
obligor, sampled according to the sector-apportioning rule of \cite{sinha2025}.

\subsubsection{Modified Contagion Feedback}

In the base model the sectoral jump upon the $n$-th default is
$\delta^{Aj}\,\ell_n$. With time-varying EAD and stochastic LGD, the realised loss
$\ell^{A_n}$ depends both on when the default occurs and on the drawn LGD, so its
scale is no longer uniform across loan types. To ensure that the contagion signal
remains comparable across repayment structures and maturities, the jump in the
intensity of sector $j$ at default time $T_n$ is normalised by the origination
principal of the defaulting obligor:
\begin{equation}
  \Delta Y^j_{T_n} \;=\; \delta^{A_n j}\,\frac{\ell^{A_n}}{P^{A_n}}.
  \label{eq:feedback_normalised}
\end{equation}
This preserves the structure of the original CIR-with-feedback dynamics while
ensuring that $\delta^{A_n j}$ retains its interpretation as a dimensionless
sensitivity parameter regardless of the repayment schedule.

\subsubsection{Parameterisation and Simulation Protocol}

The sectoral intensity dynamics follow the CIR-with-feedback specification of
\cite{sinha2025} without modification. Portfolio size, sector-weight configurations
(Concentrated, Balanced, Mixed), and idiosyncratic factor draws are held at the
values reported in that paper ($N = 1000$, $J = 3$, $X^A \sim \mathcal{U}[0.01,
0.03]$), so that all departures from the benchmark results are attributable solely to
the introduction of time-varying EAD and stochastic LGD.

Amortisation parameters are set uniformly across the portfolio: principal $P = 10$, maturity $T = 1$,
annual contractual rate $r = 0.12$, and $N_p = 12$ payment periods per year.
Recovery parameters are calibrated as described above, yielding
$(\alpha, \beta) \approx (4.5, 5.5)$. All results are based on 2{,}500 Monte Carlo
replications using the asymptotically exact acceptance--rejection algorithm of
\cite{giesecke2011}, adapted to evaluate time-dependent EAD at each accepted default
time and to draw an independent Beta LGD for each default event.

% ============================================================
%  AMORTISATION WRITE-UP  —  SECTION 2
%  Tail Risk and Expected Shortfall
% ============================================================

\section{Tail Risk and Expected Shortfall}
\label{sec:tail}

\subsection{Mechanism}

To understand why repayment structure affects the tail of the loss distribution, it
is useful to begin with the loss-generating mechanism directly. From
\eqref{eq:loss_process}, the portfolio loss at maturity is
\begin{equation}
  L_T \;=\; \sum_{n=1}^{N(T)} \mathrm{LGD}^{A_n} \cdot E^{A_n}(T_n),
  \label{eq:loss_mechanism}
\end{equation}
where $N(T)$ is the number of defaults in $[0,T]$ and $T_n$ is the time of the
$n$-th default. Conditional on the intensity path $\{Y^j_t\}$, defaults cluster
towards periods of elevated systemic intensity. Under the CIR-with-feedback dynamics
of \eqref{eq:base_intensity}, each default triggers a jump $\Delta Y^j = \delta^{Aj}
\ell^A / P^A$ in the sectoral intensities, making subsequent defaults more likely and
generating the well-documented right-skew of credit loss distributions. Crucially,
the contribution of the $n$-th default to $L_T$ is $\mathrm{LGD}^{A_n} \cdot
E^{A_n}(T_n)$: the systemic intensity and the EAD are evaluated at the
\emph{same time} $T_n$. This temporal alignment is the key to understanding the
differences across repayment structures.

For a Bullet obligor, $E_{\mathrm{B}}(T_n) = P^A$ regardless of when the default
arrives, so the full principal is always at risk. For a negatively amortising
obligor, $E_{\mathrm{N}}(T_n) = P^A(1 + r_\Delta)^{n(T_n)} > P^A$, so late defaults
--- which are also the most likely under contagion-amplified intensity --- carry an
exposure that exceeds the original notional. In both cases, maximum loss coincides
with the period of peak systemic stress. By contrast, for Linear and French
obligors the EAD is a declining function of $T_n$, so that a late-arriving,
contagion-amplified default is partially offset by the exposure reduction that
amortisation has already delivered. The speed and convexity of this decline, as
characterised by the cumulative exposure mass in Section~\ref{sec:setup}, directly
determines how much of this offset is realised.

\subsection{Risk Metrics under a Balanced Sector Allocation}

We begin by isolating the effect of repayment structure, fixing the sector-weight
configuration to the Balanced case ($w^{Aj} = 1/J$ for all $A, j$) and the
contagion parameter to $c = 0.01$. Table~\ref{tab:risk_metrics} reports the
resulting risk metrics for a portfolio of $N = 1{,}000$ obligors at $T = 1$,
under the stochastic Beta LGD framework with $(\alpha, \beta) \approx (4.5, 5.5)$.

\begin{table}[hbt!]
\centering
\caption{Portfolio risk metrics across amortisation schedules under a Balanced
sector allocation ($N = 1{,}000$ obligors, $T = 1$, $c = 0.01$).}
\label{tab:risk_metrics}
\begin{tabular}{lccccc}
\toprule
\textbf{Schedule} & \textbf{Mean Loss} & \textbf{Std Dev}
  & \textbf{VaR}$_{95\%}$ & \textbf{ES}$_{95\%}$ & \textbf{Excess ES} \\
\midrule
Bullet                & 445.16 & 92.11 & 603.68 & 653.19 & 49.51 \\
French                & 192.02 & 39.74 & 260.42 & 282.88 & 22.46 \\
Linear (Italian)      & 187.92 & 39.41 & 254.87 & 277.08 & 22.21 \\
\bottomrule
\end{tabular}
\end{table}

The results establish a clear ordering. The Bullet schedule generates a mean loss
of 445.16 with a 95\% Expected Shortfall (ES) of 653.19, more than double the ES
of either amortising schedule. The Linear (Italian) contract attains the lowest ES
at 277.08, with the French annuity structure following closely at 282.88. The
moderate gap between the two amortising schedules --- approximately $2\%$ in ES
terms --- reflects the difference in their exposure profiles identified in
Section~\ref{sec:cumulative_mass}: the French schedule maintains a higher
outstanding balance in the early-to-middle portion of the horizon, where the
probability of contagion-amplified defaults is concentrated, whereas the Linear
schedule has already returned a proportionate share of principal by that point.

The \emph{Excess ES}, defined as $\text{ES}_{95\%} - \text{VaR}_{95\%}$, provides a
measure of tail thickness conditional on a tail event having already occurred. Under
the Bullet schedule, Excess ES reaches 49.51 --- more than double that of the
amortising schedules (22.46 and 22.21 respectively). This disproportionate expansion
indicates that tail events under the Bullet are not merely more frequent but
materially more severe: once the 95th percentile threshold is crossed, losses
continue to accumulate substantially beyond it.

\subsection{Loss Distributions across Sectoral Configurations}

The analysis in Table~\ref{tab:risk_metrics} holds the sectoral weight structure
fixed. We now allow it to vary across the four configurations introduced in
\cite{sinha2025}: Balanced ($w^{Aj} = 1/3$), Concentrated ($w^{A,1} = 0.7$,
residual equally split), Mixed ($w^{A,1} = 0.5$, $w^{A,2} = 0.3$, $w^{A,3} =
0.2$), and Random (Dirichlet-drawn weights). Figures~\ref{fig:loss_balanced}--\ref{fig:loss_random}
present the simulated loss distributions for the three completed schedules (Bullet,
French, Linear) under each sectoral configuration. Across all four configurations, the
qualitative ordering from Table~\ref{tab:risk_metrics} is preserved: the Bullet
distribution consistently exhibits the heaviest right tail and the widest spread,
with the two amortising schedules forming a tightly clustered pair well below it.

Within this stable ordering, sector concentration acts as an amplifier. Comparing
the Concentrated configuration to the Balanced, the Bullet tail extends considerably
further to the right, whilst the amortising schedules, though also shifting outward,
do so proportionately less. This asymmetric amplification reflects the interaction
between weight concentration and the contagion feedback of \eqref{eq:feedback_normalised}:
a dominant sector absorbs a larger jump $\delta^{Aj} \ell^A / P^A$ at each default,
which in turn elevates the intensities of the obligors most exposed to that sector
--- precisely those who carry the highest weight in a concentrated portfolio. Late
defaults in a Bullet portfolio therefore arrive not only
into an already-stressed intensity environment but with full exposure,
compounding the tail loss beyond what would occur in a diversified setting.

The loss density plots also reveal a shift in distributional shape. Under the
Balanced configuration, all three schedules produce approximately unimodal,
right-skewed densities. Under the Concentrated configuration, the Bullet
distribution develops a notably heavier and longer right tail, consistent with the
heavy-tail amplification reported in \cite{sinha2025} for the concentrated sector
case. The French and Linear distributions remain largely unimodal and relatively
tight, confirming that the protective effect of continuous amortisation is robust to
sector concentration.

\begin{figure}[hbt!]
  \centering
  \includegraphics[width=\textwidth]{figures/loss_balanced.png}
  \caption{Simulated loss distributions (histogram and kernel density estimate) for
  the Bullet, French, and Linear amortisation schedules under a \textit{Balanced}
  sector allocation ($w^{Aj} = 1/3$, $c = 0.01$, $N = 1{,}000$, 2{,}500 paths).}
  \label{fig:loss_balanced}
\end{figure}

\begin{figure}[hbt!]
  \centering
  \includegraphics[width=\textwidth]{figures/loss_conc.png}
  \caption{Loss distributions under a \textit{Concentrated} sector allocation
  ($w^{A,1} = 0.7$). Sector concentration amplifies the right tail of the Bullet
  schedule disproportionately relative to the amortising schedules.}
  \label{fig:loss_concentrated}
\end{figure}

\begin{figure}[hbt!]
  \centering
  \includegraphics[width=\textwidth]{figures/loss_mixed.png}
  \caption{Loss distributions under a \textit{Mixed} sector allocation
  ($w^{A,1} = 0.5$, $w^{A,2} = 0.3$, $w^{A,3} = 0.2$). The mixed case
  displays intermediate tail behaviour, consistent with partial diversification.}
  \label{fig:loss_mixed}
\end{figure}

\begin{figure}[hbt!]
  \centering
  \includegraphics[width=\textwidth]{figures/loss_random.png}
  \caption{Loss distributions under a \textit{Random} (Dirichlet-drawn) sector
  allocation. The random configuration spans a range of concentration levels and
  its loss distribution lies between the Balanced and Concentrated cases.}
  \label{fig:loss_random}
\end{figure}

\subsection{Synthesis: ES and Excess ES across Schedules and Sectors}

Figure~\ref{fig:es_summary} consolidates the two dimensions of variation ---
repayment structure and sectoral allocation --- into a single summary of ES$_{95\%}$
and Excess ES across the three primary sector configurations. Several conclusions
emerge from this joint view.

First, repayment structure is the dominant determinant of the level of ES. Across
all sectoral configurations, the Bullet schedule consistently registers an ES
approximately 2.3 times that of the Linear schedule and 2.3 times that of the
French schedule, a ratio that is remarkably stable across sector configurations.
This stability indicates that the protective effect of amortisation is not contingent
on any particular degree of sectoral diversification.

Second, sector concentration governs the \emph{sensitivity} of ES to the repayment
schedule. In the Concentrated configuration, the absolute gap between Bullet ES and
Linear ES is wider than in the Balanced case, confirming the amplification mechanism
discussed above. The Mixed configuration registers intermediate gaps throughout,
consistent with its partial concentration.

Third, the Excess ES panel reveals that the difference in tail \emph{thickness} ---
as opposed to tail \emph{level} --- is also driven primarily by the repayment
schedule. Even under the Balanced configuration, where sector diversification is
maximal, the Bullet Excess ES is approximately double that of the amortising
schedules. This implies that the tail-thinning effect of amortisation is structural
and cannot be recovered by diversification alone.

\begin{figure}[hbt!]
  \centering
  \includegraphics[width=0.8\textwidth]{figures/es-summary.png}
  \caption{ES$_{95\%}$ (upper panel) and Excess ES (lower panel) by loan type across
  the Balanced, Concentrated, and Mixed sector configurations.}
  \label{fig:es_summary}
\end{figure}

% ============================================================
%  AMORTISATION WRITE-UP  —  SECTION 3
%  Survival Probability and Time to Systemic Stress
% ============================================================

\section{Survival Probability and Time to Systemic Stress}
\label{sec:survival}

\subsection{Motivation and Setup}

Expected Shortfall characterises the \emph{magnitude} of tail losses at the
terminal horizon $T$, but it is silent on the \emph{temporal path} by which the
portfolio arrives at a stressed state. From a risk management perspective, the
speed with which a portfolio transitions into a systemic stress regime is at least
as important as the eventual loss level: a portfolio that reaches critical intensity
early in its life affords almost no time for supervisory intervention, whilst one
that defers the onset of stress allows hedging and deleveraging to act. The first
passage time to a stress threshold is the natural object for capturing this
dimension.

We define the aggregate systemic intensity at time $t$ as
\begin{equation}
  \mathcal{Y}_t \;=\; \sum_{j=1}^{J} Y^j_t,
  \label{eq:aggregate_intensity}
\end{equation}
and the first passage time (FPT) to a critical stress level $y^\star$ as
\begin{equation}
  \tau^\star \;=\; \inf\!\left\{\, t > 0 \;\middle|\; \mathcal{Y}_t \geq y^\star \,\right\},
  \label{eq:fpt}
\end{equation}
with $\tau^\star = +\infty$ on paths where $\mathcal{Y}_t < y^\star$ for all
$t \in [0,T]$. The survival probability at time $t$ is then
\begin{equation}
  S(t) \;=\; \mathbb{P}(\tau^\star > t),
  \label{eq:survival}
\end{equation}
which equals one at $t = 0$ and declines monotonically. Paths for which
$\tau^\star = +\infty$ contribute to a positive floor $S(T) = \mathbb{P}(\tau^\star >
T) = 1 - \mathbb{P}(\tau^\star \leq T) > 0$: the system never crosses the threshold
within the observed horizon. This censoring is an intrinsic feature of the problem
and must be treated carefully in estimation.

\subsection{Threshold Definitions}

To ensure robustness of conclusions across different notions of systemic stress, we
evaluate \eqref{eq:fpt} under two alternative threshold specifications.

\paragraph{Threshold~1 — Dynamic Quantile.}
The threshold $y^\star_{\mathrm{Q}}$ is set to the 95th percentile of the
empirical distribution of $\max_{t \in [0,T]} \mathcal{Y}_t$ across Monte Carlo
paths generated by a baseline portfolio in which LGD is uniform and EAD is
constant (i.e.\ no amortisation). Formally,
\begin{equation}
  y^\star_{\mathrm{Q}} \;=\; \hat{F}^{-1}_{Y_{\max}}\!(0.95),
  \label{eq:threshold_quantile}
\end{equation}
where $\hat{F}_{Y_{\max}}$ is the empirical CDF of peak intensity under the
baseline. This calibration ensures that $y^\star_{\mathrm{Q}}$ represents a
genuinely extreme but historically grounded stress level, rather than an
arbitrarily chosen scalar.

\paragraph{Threshold~2 — Deterministic Structural.}
The threshold $y^\star_{\mathrm{D}}$ is set as a fixed proportion of the maximum
theoretical baseline intensity:
\begin{equation}
  y^\star_{\mathrm{D}} \;=\; 0.75 \times \sum_{j=1}^{J} \lambda^j_{\max},
  \label{eq:threshold_det}
\end{equation}
where $\lambda^j_{\max}$ is the upper level used in the acceptance--rejection
algorithm of \cite{giesecke2011} for sector $j$. This threshold is independent of
any particular Monte Carlo sample, making it transparent and replicable across
parameterisations.

The two thresholds probe different aspects of systemic stress. The Dynamic Quantile
threshold is data-adaptive and tied to extreme realisations of aggregate intensity
under the \emph{base} model, so crossing it under any amortised schedule constitutes
a genuinely anomalous event. The Deterministic threshold provides a fixed structural
benchmark against which the stabilising effect of amortisation can be assessed
without dependence on baseline simulations.

\subsection{Empirical Estimation}

For each repayment schedule and sector configuration, we simulate $M = 2{,}500$
independent paths of the full system. Each path yields a first passage time
$\tau^\star_m \in (0, +\infty]$. The empirical survival function is estimated as
\begin{equation}
  \hat{S}(t) \;=\; 1 \;-\; \frac{1}{M} \sum_{m=1}^{M} \mathbf{1}_{\{\tau^\star_m \leq t\}},
  \label{eq:empirical_survival}
\end{equation}
which is the standard empirical CDF complement and accounts correctly for censored
paths: a path with $\tau^\star_m = +\infty$ contributes zero to the sum for all
finite $t$, so the estimator converges to $\hat{S}(T) = 1 - \hat{\mathbb{P}}(\tau^\star \leq T)$
at the terminal date. The survival curve is plotted as a step function, consistent
with the discrete-time structure of the simulation.

Two summary statistics are extracted for each configuration:
\begin{itemize}
  \item \textbf{Probability of first passage:} $\hat{p} = 1 - \hat{S}(T)$, the
    fraction of paths that cross $y^\star$ within $[0, T]$.
  \item \textbf{Mean FPT (conditional):} $\hat{\mu} = \mathrm{mean}\{\tau^\star_m :
    \tau^\star_m < \infty\}$, the average time to threshold crossing among paths
    that do cross, analogous to a conditional mean survival time in the presence
    of censoring.
\end{itemize}
The conditional mean FPT is the more informative summary statistic when a
substantial fraction of paths never cross the threshold, as it measures how quickly
systemic stress materialises in the scenarios where it does.

\subsection{Mechanism: Why Amortisation Delays the First Passage}
\label{sec:fpt_mechanism}

The connection between amortisation and the FPT runs directly through the normalised
feedback of \eqref{eq:feedback_normalised}. Each default event at time $T_n$
generates a jump in the sectoral intensities of magnitude
\begin{equation}
  \Delta Y^j_{T_n} \;=\; \delta^{A_n j}\,\frac{\mathrm{LGD}^{A_n} \cdot E^{A_n}(T_n)}{P^{A_n}}.
  \label{eq:jump_fpt}
\end{equation}
For a Bullet obligor, $E^{A_n}(T_n) = P^{A_n}$ for all $T_n < T$, so the
normalised jump is simply $\delta^{A_n j} \cdot \mathrm{LGD}^{A_n}$, independent
of when the default occurs. For a Linear or French obligor, $E^{A_n}(T_n)/P^{A_n}
< 1$ for any $T_n > 0$, and this ratio declines monotonically with $T_n$. It
follows that, for any fixed realisation of the LGD, an amortising obligor generates
a strictly smaller intensity jump than a Bullet obligor defaulting at the same time.
This attenuated jump slows the self-reinforcing feedback loop that drives
$\mathcal{Y}_t$ towards $y^\star$, delaying and in some paths entirely preventing
the threshold crossing.

The attenuation is most pronounced for defaults that arrive late in the horizon,
when contagion has already built up and the portfolio is closest to the threshold.
For the Linear schedule, $E_{\mathrm{L}}(t)/P = 1 - t/T$, so a default at
$t = 0.8T$ carries only $20\%$ of the full principal as contagion weight. For the
French schedule, $E_{\mathrm{F}}(t)/P$ remains closer to unity throughout (reflecting
the convex EAD profile), so the attenuation is less pronounced than for Linear
in the early-to-middle part of the horizon but comparable near maturity. This
difference mirrors the finding in Section~\ref{sec:tail} that the Linear schedule
provides marginally more tail protection than the French under the same contagion
parameter, and it implies that the Linear schedule should also exhibit a slightly
lower probability of first passage and a longer conditional mean FPT.

\subsection{Results}

\subsubsection{Balanced Sector Allocation}

Figures~\ref{fig:survival_bal_q} and~\ref{fig:survival_bal_d} present the empirical
survival curves under the Balanced sector configuration for the Dynamic Quantile
and Deterministic thresholds respectively.

Under Threshold~1, the Bullet curve declines sharply from the outset, reflecting the
high probability that even early defaults generate sufficient contagion to push
$\mathcal{Y}_t$ into the tail of the baseline intensity distribution. The amortising
schedules (Linear and French) display a markedly slower decline, with a non-trivial
fraction of paths surviving to maturity without crossing the threshold — evidenced by
$\hat{S}(T) > 0$ at the right edge of the plot. The gap between the Bullet and
amortising curves is greatest in the mid-horizon region, where contagion has had
sufficient time to accumulate under the Bullet but amortisation has already
meaningfully reduced the feedback magnitude under the amortising schedules.

Under Threshold~2, the qualitative ordering is identical. Because the Deterministic
threshold is fixed independently of sample variation in peak intensity, the survival
curves are smoother and the floor $\hat{S}(T)$ is more stable across simulation
runs. The Linear schedule consistently registers a slightly higher survival
probability than the French throughout the horizon, consistent with the mechanism
described in Section~\ref{sec:fpt_mechanism}: the earlier onset of EAD reduction
under the Linear schedule attenuates contagion jumps from a larger fraction of
potential default events.

\begin{figure}[hbt!]
  \centering
  \includegraphics[width=0.8\textwidth]{figures/survival_bal_quantile.png}
  \caption{Empirical survival probability $\hat{S}(t) = \hat{\mathbb{P}}(\tau^\star
  > t)$ under a \textit{Balanced} sector allocation and the \textit{Dynamic Quantile}
  threshold ($y^\star_{\mathrm{Q}} = \hat{F}^{-1}_{Y_{\max}}(0.95)$). Step
  functions are plotted over 2{,}500 Monte Carlo paths. The Bullet schedule
  crosses the threshold markedly earlier than the amortising schedules; a
  fraction of Linear and French paths never cross within $[0,T]$.}
  \label{fig:survival_bal_q}
\end{figure}

\begin{figure}[hbt!]
  \centering
  \includegraphics[width=0.8\textwidth]{figures/survival_bal_det.png}
  \caption{Empirical survival curves under a \textit{Balanced} sector allocation
  and the \textit{Deterministic} threshold ($y^\star_{\mathrm{D}} = 0.75\,
  \sum_j \lambda^j_{\max}$). The qualitative ordering is preserved; the
  Deterministic threshold yields smoother step functions owing to its
  independence from sample variation in peak intensity.}
  \label{fig:survival_bal_d}
\end{figure}

\subsubsection{Concentrated Sector Allocation}

Figures~\ref{fig:survival_con_q} and~\ref{fig:survival_con_d} repeat the analysis
under the Concentrated configuration ($w^{A,1} = 0.7$).

Concentration has two compounding effects on the FPT. First, it raises the baseline
level of sectoral intensity: with the majority of obligors co-exposed to a single
dominant sector, any default within that sector generates a large concentrated jump
in $Y^1_t$, accelerating $\mathcal{Y}_t$ towards $y^\star$. Second, it reduces the
stabilising effect of amortisation. Because the dominant sector absorbs a fraction
$w^{A,1} = 0.7$ of every contagion jump \eqref{eq:jump_fpt}, and because the
Bullet schedule keeps this jump at its maximum level for all $T_n \in [0,T)$, the
concentrated Bullet portfolio reaches the threshold with substantially higher
probability and at substantially earlier times than in the Balanced case. The
amortising schedules are also accelerated under concentration, but proportionately
less so: the attenuation of their contagion jumps through declining EAD still
operates, even if the dominant sector amplifies those jumps before attenuation.

Consequently, under the Concentrated configuration the survival curves of all three
schedules shift towards earlier first passage times and lower terminal survival
probabilities $\hat{S}(T)$. The gap between the Bullet and the amortising curves
remains clearly visible and actually widens in absolute probability terms, confirming
that the protective effect of amortisation is not eroded by concentration.

\begin{figure}[hbt!]
  \centering
  \includegraphics[width=0.8\textwidth]{figures/survival_con_quantile.png}
  \caption{Empirical survival curves under a \textit{Concentrated} sector allocation
  and the \textit{Dynamic Quantile} threshold. Concentration accelerates first
  passage across all schedules; the Bullet schedule reaches $\hat{S}(t) \approx 0$
  earlier than in the Balanced case, whilst the amortising schedules retain a
  higher terminal survival probability.}
  \label{fig:survival_con_q}
\end{figure}

\begin{figure}[hbt!]
  \centering
  \includegraphics[width=0.8\textwidth]{figures/survival_con_det.png}
  \caption{Empirical survival curves under a \textit{Concentrated} sector allocation
  and the \textit{Deterministic} threshold. The widening of the gap between Bullet
  and amortising schedules under concentration confirms that the stabilising effect
  of amortisation is robust to, and in fact strengthened by, increased sectoral
  concentration.}
  \label{fig:survival_con_d}
\end{figure}

\subsubsection{Summary Metrics}

Table~\ref{tab:fpt_metrics} reports the two scalar FPT statistics for each
combination of schedule, sector configuration, and threshold. Several patterns
deserve emphasis.

The probability of first passage $\hat{p}$ is uniformly highest for the Bullet
schedule and lowest for the Linear schedule, across all four combinations. Under the
Balanced configuration the gap between Bullet and Linear first-passage probabilities
is substantial under both thresholds, confirming that a meaningful fraction of
amortising-portfolio paths avoid systemic stress entirely within the horizon. Under
the Concentrated configuration, first-passage probabilities rise for all schedules,
but the Bullet schedule consistently reaches values close to 1 whilst the amortising
schedules retain a lower probability, reflecting the attenuated contagion mechanism.

The conditional mean FPT $\hat{\mu}$ is strictly ordered: Linear $>$ French $>$
Bullet in all cases. This ordering is consistent with the EAD-attenuation mechanism:
the Linear schedule reduces EAD at a constant rate from the outset, so its
contagion-dampening effect is strongest in the early and middle portions of the
horizon where crossing is most likely; the French schedule delivers comparable
attenuation only later, when the annuity structure's accelerating amortisation
catches up. Under concentration, all conditional mean FPTs shorten, but the
relative ordering and the magnitude of the Linear--Bullet gap are preserved.

\begin{table}[hbt!]
\centering
\caption{First passage time statistics across amortisation schedules, sector
configurations, and threshold types ($N = 1{,}000$, $T = 2$, $c = 0.01$,
2{,}500 Monte Carlo paths). $\hat{p}$: probability of crossing $y^\star$ within
$[0,T]$. $\hat{\mu}$: mean FPT conditional on finite crossing.}
\label{tab:fpt_metrics}
\setlength{\tabcolsep}{6pt}
\begin{tabular}{llcccc}
\toprule
& & \multicolumn{2}{c}{\textbf{Dynamic Quantile}} &
    \multicolumn{2}{c}{\textbf{Deterministic}} \\
\cmidrule(lr){3-4}\cmidrule(lr){5-6}
\textbf{Sector} & \textbf{Schedule} & $\hat{p}$ & $\hat{\mu}$ & $\hat{p}$ & $\hat{\mu}$ \\
\midrule
\multirow{3}{*}{Balanced}
  & Bullet           & 0.880 & 1.394 & 0.985 & 1.072 \\
  & French           & 0.012 & 1.170 & 0.174 & 1.057 \\
  & Linear (Italian) & 0.005 & 1.130 & 0.129 & 1.071 \\
\midrule
\multirow{3}{*}{Concentrated}
  & Bullet           & 1.000 & 0.857 & 1.000 & 0.686 \\
  & French           & 0.482 & 1.073 & 0.854 & 0.878 \\
  & Linear (Italian) & 0.420 & 1.087 & 0.799 & 0.879 \\
\bottomrule
\end{tabular}
\end{table}

% ============================================================
%  AMORTISATION WRITE-UP  —  SECTION 4
%  Contagion Amplification and Default Clustering
% ============================================================

\section{Contagion Amplification and Default Clustering}
\label{sec:contagion}

\subsection{From Tail Losses to Intensity Dynamics}

Sections~\ref{sec:tail} and~\ref{sec:survival} established that repayment structure
governs both the terminal loss distribution and the speed of transition to systemic
stress. The present section examines the \emph{mechanism} by which these differences
arise, decomposing the contagion channel into its intensity and clustering components.

The bridge between EAD and contagion is the normalised feedback
of~\eqref{eq:feedback_normalised}. Each default at time $T_n$ inflicts a jump
\begin{equation}
  \Delta\mathcal{Y}_{T_n} \;=\; \sum_{j=1}^{J} \delta^{A_n j}\,
  \frac{\mathrm{LGD}^{A_n} \cdot E^{A_n}(T_n)}{P^{A_n}}
  \label{eq:agg_jump}
\end{equation}
on the aggregate intensity $\mathcal{Y}_t = \sum_j Y^j_t$. For a Bullet obligor,
$E^{A_n}(T_n)/P^{A_n} = 1$ throughout, so every default delivers the maximum
possible feedback regardless of timing. For a negatively amortising obligor, this
ratio exceeds unity, so defaults --- which are themselves more likely when intensity
is elevated --- deliver \emph{super-par} jumps, creating a doubly reinforcing
feedback loop. For Linear and French obligors the ratio is strictly less than one
for any $T_n > 0$ and monotonically declining, attenuating each contagion jump
relative to the Bullet benchmark.

This attenuation has a compounding character. Early defaults reduce $\mathcal{Y}_t$
below the Bullet trajectory, which in turn reduces the probability of subsequent
defaults, which in turn reduces the future feedback into $\mathcal{Y}_t$. The
amortisation schedule therefore acts as a \emph{negative feedback brake} on the
self-exciting intensity process, whilst the Bullet and Negative Amortisation
schedules remove this brake entirely or reverse it. The consequence is not merely a
shift in the loss distribution --- it is a qualitative change in the dynamics of
$\mathcal{Y}_t$ under stress, which we now quantify.

\subsection{Peak Systemic Intensity}

We characterise the severity of intensity excursions via the path-level maximum
\begin{equation}
  \mathcal{Y}_{\max} \;=\; \sup_{t \in [0,T]} \mathcal{Y}_t,
  \label{eq:ymax}
\end{equation}
which measures the most extreme stress state reached during each simulated scenario.
Unlike the first passage time of Section~\ref{sec:survival}, which records
\emph{when} a threshold is crossed, $\mathcal{Y}_{\max}$ records \emph{how far}
the system departs from its baseline level over the full horizon.

Table~\ref{tab:clustering_metrics} reports the 99th percentile of
$\mathcal{Y}_{\max}$ across 2{,}500 paths under the Balanced sector configuration.
The Bullet schedule attains a 99th-percentile peak of $1.090$, more than double the
corresponding figures for the Linear ($0.529$) and French ($0.560$) schedules. The
ratio of $\approx 2\times$ in extreme peak intensity is consistent with the $\approx
2.3\times$ ratio in ES$_{95\%}$ reported in Table~\ref{tab:risk_metrics}: both
reflect the same underlying difference in contagion jump magnitudes. The slight
elevation of the French 99th-percentile relative to Linear ($0.560$ vs $0.529$)
mirrors the finding from Sections~\ref{sec:cumulative_mass} and~\ref{sec:survival}
that the French schedule maintains higher outstanding balances in the early-to-middle
portion of the horizon, where the intensity is most likely to be in a critical
growth phase.

Figure~\ref{fig:ymax_scatter} presents a scatter plot of $\mathcal{Y}_{\max}$
against total portfolio loss $L_T$ across all simulated paths and schedules. The
relationship is strongly positive in all cases, confirming that the loss level is a
direct consequence of intensity excursion severity rather than an artefact of the
LGD draw or the EAD profile in isolation. Crucially, the clouds for the Bullet and
Negative Amortisation schedules are shifted both rightward (higher $L_T$) and
upward (higher $\mathcal{Y}_{\max}$) relative to the amortising schedules, and the
two dimensions of shift are proportionally aligned: larger intensity peaks produce
larger total losses because the firms surviving to the peak are still exposed to
near-full or super-par principal. Under the amortising schedules the clouds are more
compressed, and the slope of the $\mathcal{Y}_{\max}$--$L_T$ relationship is
shallower: even when the intensity does spike, the reduced EAD limits the loss that
the spike can generate. This compression is the geometric expression of the
negative-feedback brake described above.

\begin{figure}[hbt!]
  \centering
  \includegraphics[width=0.8\textwidth]{figures/ymax_scatter.png}
  \caption{Scatter plot of peak systemic intensity $\mathcal{Y}_{\max}$ versus
  total portfolio loss $L_T$ across 2{,}500 Monte Carlo paths, by amortisation
  schedule (Balanced sector configuration, $c = 0.01$).}
  \label{fig:ymax_scatter}
\end{figure}

\subsection{Default Clustering and the Amplification Factor}

Peak intensity characterises aggregate system stress, but does not directly
reveal how defaults \emph{cluster} within a scenario. To isolate the clustering
effect we compare the unconditional expected number of defaults $\mathbb{E}[N(T)]$
with its conditional counterpart in the loss tail:
\begin{equation}
  \mathbb{E}\!\left[\,N(T) \;\middle|\; L_T \geq \mathrm{VaR}_{95\%}\right],
  \label{eq:conditional_defaults}
\end{equation}
where $\mathrm{VaR}_{95\%}$ is the schedule-specific 95th percentile loss. We
define the \emph{Amplification Factor} as the ratio of these two expectations:
\begin{equation}
  \mathcal{A} \;=\; \frac{\mathbb{E}[N(T) \mid L_T \geq \mathrm{VaR}_{95\%}]}
    {\mathbb{E}[N(T)]}.
  \label{eq:amplification}
\end{equation}
$\mathcal{A} > 1$ indicates that tail loss scenarios are characterised by more
defaults than a typical scenario --- the signature of genuine default clustering
rather than a single large loss event. $\mathcal{A}$ is estimated empirically by
replacing expectations with sample means over the relevant subsets of Monte Carlo
paths.

The results in Table~\ref{tab:clustering_metrics} reveal a striking pattern: the
Amplification Factor is essentially stable across all three schedules, ranging
narrowly from $1.328$ to $1.356$. This near-invariance is not coincidental. The
Amplification Factor measures the \emph{relative} intensification of defaults in
the tail and, under the affine jump-diffusion structure, this relative intensification
is governed primarily by the contagion parameter $c$. Because $c = 0.01$ is held
fixed across schedules, the multiplicative clustering behaviour of the process is
held fixed; only its \emph{scale} changes with the EAD profile. The amortisation
schedule is therefore not a clustering parameter --- it does not alter \emph{how
much} more defaults cluster in the tail relative to the unconditional mean --- but
rather an \emph{exposure scale parameter} that determines the loss consequence of
each default in the cluster.

This distinction has a precise implication. Because $\mathcal{A}$ is stable whilst
$\mathbb{E}[N(T) \mid \text{Tail}]$ differs substantially across schedules --- from
247.82 under Linear to 358.69 under Bullet, a difference of more than 110 defaults
--- the gap is entirely attributable to the difference in $\mathbb{E}[N(T)]$
itself: the Bullet portfolio simply experiences more defaults in expectation, because
elevated EAD amplifies the contagion jump at each event, raising the subsequent
intensity and accelerating future defaults. The higher conditional tail count is
thus a consequence of the same positive feedback loop that elevates $\mathcal{Y}_{\max}$,
not an independent clustering effect. Amortisation suppresses the unconditional
default rate, and this suppression scales proportionately into the tail.

\begin{table}[hbt!]
\centering
\caption{Default clustering and peak intensity metrics under a Balanced sector
allocation ($N = 1{,}000$, $T = 1$, $c = 0.01$,
$\mathrm{LGD} \sim \mathrm{Beta}(4.5,\,5.5)$, 2{,}500 Monte Carlo paths).
$\mathcal{Y}_{\max}^{(99)}$: 99th percentile of path-level peak aggregate
intensity. $\hat{\mathbb{E}}[N]$: mean number of defaults per path.
$\hat{\mathbb{E}}[N \mid \mathrm{Tail}]$: mean defaults conditional on
$L_T \geq \widehat{\mathrm{VaR}}_{95\%}$.
$\hat{\mathcal{A}}$: empirical Amplification Factor.}
\label{tab:clustering_metrics}
\setlength{\tabcolsep}{8pt}
\begin{tabular}{lcccc}
\toprule
\textbf{Schedule}
  & $\mathcal{Y}_{\max}^{(99)}$
  & $\hat{\mathbb{E}}[N]$
  & $\hat{\mathbb{E}}[N \mid \mathrm{Tail}]$
  & $\hat{\mathcal{A}}$ \\
\midrule
Bullet                & 1.090 & 265.47 & 358.69 & 1.351 \\
French                & 0.560 & 191.15 & 259.19 & 1.356 \\
Linear (Italian)      & 0.529 & 186.63 & 247.82 & 1.328 \\
\bottomrule
\end{tabular}
\end{table}

\subsection{Joint Implications}

Taken together, the peak intensity and clustering results deliver two complementary
insights. First, repayment structure is a first-order determinant of the
\emph{level} of systemic stress: the Bullet schedule generates intensity excursions
roughly twice as severe as the amortising schedules (as measured by
$\mathcal{Y}_{\max}^{(99)}$), driving a proportionate increase in both the mean and
tail of the loss distribution. Second, the \emph{relative} clustering structure ---
how much worse the tail is compared to the unconditional expectation --- is governed
by the contagion parameter $c$ and is approximately invariant to the EAD profile.
The practical implication is direct: a regulator or risk manager seeking to reduce
tail losses should treat the amortisation schedule as a lever of comparable
importance to the contagion sensitivity $c$. Tightening $c$ through collateral
requirements or exposure limits reduces the Amplification Factor; mandating earlier
principal repayment reduces the \emph{scale} of every loss in the distribution,
including those in the tail, without requiring any change to the contagion
structure.

% ============================================================
%  AMORTISATION WRITE-UP  —  SECTION 5
%  Conclusion and Theoretical Implications
% ============================================================

\section{Conclusion and Theoretical Implications}
\label{sec:conclusion}

\subsection{Summary of Findings}

This paper has extended the sector-apportioned intensity framework of \cite{sinha2025}
by replacing the static, uniform loss draw with contractually specified, time-varying
EAD schedules and a Beta-distributed stochastic LGD. The four canonical amortisation
structures --- Bullet, Linear (Italian), French, and Negative Amortisation --- were
evaluated across sectoral concentration regimes and two systemic stress thresholds,
using 2{,}500 Monte Carlo paths of the asymptotically exact acceptance--rejection
algorithm of \cite{giesecke2011}.

Three quantitative findings stand out. First, repayment structure is the dominant
determinant of tail loss severity. The Bullet schedule generates a 95\% Expected
Shortfall approximately $2.3$ times that of the Linear schedule under a Balanced
sector allocation, a ratio that is stable across concentration regimes. Second,
early principal repayment materially extends the system's survival time: Linear and
French portfolios exhibit substantially lower first-passage probabilities and longer
conditional mean passage times to the systemic stress threshold, regardless of which
threshold specification is used. Third, the Amplification Factor --- the ratio of
conditional-tail to unconditional expected defaults --- is approximately invariant to
the repayment schedule, ranging from $1.33$ to $1.36$ across all three completed
schedules. This invariance reveals that amortisation does not alter the clustering
structure of the default process, which is governed by the contagion parameter $c$;
it acts instead as an exposure scale parameter that attenuates every loss in the
distribution, including those in the tail, proportionately and unconditionally.

\subsection{A Theoretical Note on the Convex Ordering of Losses}

Beyond the simulation evidence, the direction of the risk ordering across repayment
structures can be established via a coupling argument rooted in the model's jump
structure. Consider two portfolios, $\mathcal{P}_A$ and $\mathcal{P}_B$, driven by
identical realisations of the Brownian motions $\{W^j_t\}$, idiosyncratic factors
$\{X^A\}$, and LGD draws $\{\mathrm{LGD}^A\}$, but with exposure schedules
satisfying
\begin{equation}
  E^A_{\mathcal{P}_A}(t) \;\leq\; E^A_{\mathcal{P}_B}(t)
  \qquad \text{for all } A \text{ and all } t \in [0,T].
  \label{eq:ead_order}
\end{equation}
Under this coupling, the first default time $T_1$ is identical across both
portfolios, since the intensity process prior to the first event is driven entirely
by the sectoral CIR dynamics and the idiosyncratic constants, which are common.
However, the contagion jump at $T_1$ satisfies
$\Delta\mathcal{Y}_{T_1}^{\mathcal{P}_B} \geq \Delta\mathcal{Y}_{T_1}^{\mathcal{P}_A}$
pathwise, because $E^{A_1}_{\mathcal{P}_B}(T_1) \geq E^{A_1}_{\mathcal{P}_A}(T_1)$
by~\eqref{eq:ead_order}. After $T_1$, the sectoral intensity of $\mathcal{P}_B$
lies at or above that of $\mathcal{P}_A$ on every path, which by the stochastic
comparison theorem for jump-diffusions \cite{ikeda1989} implies
\begin{equation}
  \mathcal{Y}_t^{\mathcal{P}_B} \;\geq_{\mathrm{st}}\; \mathcal{Y}_t^{\mathcal{P}_A}
  \qquad \text{for all } t > T_1,
  \label{eq:intensity_order}
\end{equation}
where $\geq_{\mathrm{st}}$ denotes first-order stochastic dominance. This ordering
propagates to all subsequent default times and loss marks: since $\mathcal{P}_B$
operates at a persistently higher intensity with persistently higher EAD, its
cumulative loss process satisfies
\begin{equation}
  L_T^{\mathcal{P}_B} \;\geq_{\mathrm{st}}\; L_T^{\mathcal{P}_A}.
  \label{eq:loss_order}
\end{equation}
First-order stochastic dominance implies convex ordering \cite{shaked2007}, so
$L_T^{\mathcal{P}_A} \leq_{cx} L_T^{\mathcal{P}_B}$. Since Expected Shortfall is a
law-invariant coherent risk measure that is monotone with respect to convex order
\cite{follmer2016}, we obtain
\begin{equation}
  \mathrm{ES}_\alpha(L_T^{\mathcal{P}_A}) \;\leq\; \mathrm{ES}_\alpha(L_T^{\mathcal{P}_B})
  \qquad \text{for all } \alpha \in (0,1).
  \label{eq:es_order}
\end{equation}
The EAD pointwise ordering~\eqref{eq:ead_order} is satisfied by the schedule
hierarchy identified empirically:
\begin{equation}
  E_{\mathrm{Linear}}(t) \;\leq\; E_{\mathrm{French}}(t) \;\leq\;
  E_{\mathrm{Bullet}}(t) \;\leq\; E_{\mathrm{Neg.\,Am.}}(t)
  \qquad \forall\, t \in (0, T),
  \label{eq:ead_hierarchy}
\end{equation}
from which~\eqref{eq:es_order} gives
$\mathrm{ES}_\alpha(L_T^{\mathrm{Linear}}) \leq \mathrm{ES}_\alpha(L_T^{\mathrm{French}})
\leq \mathrm{ES}_\alpha(L_T^{\mathrm{Bullet}}) \leq \mathrm{ES}_\alpha(L_T^{\mathrm{Neg.\,Am.}})$
for all $\alpha$. The heavy tails observed in the Monte Carlo results are therefore
not artefacts of the chosen parameterisation; they are intrinsic mathematical
properties of the contagion feedback loop when principal repayment is deferred.

We note two caveats. First, the coupling argument relies on the pathwise ordering
of the intensity holding between successive default events, which is guaranteed under
the CIR specification by the comparison theorem but may require separate verification
if the sectoral dynamics are replaced by a more general diffusion. Second, the
inequality in~\eqref{eq:ead_hierarchy} is strict for all $t \in (0, T)$ but
collapses at $t = T$ for the Bullet schedule ($E_{\mathrm{B}}(T) = 0$), so the
ordering is a property of the interior of the horizon rather than its endpoint.

\subsection{Implications for Risk Management and Regulation}

These results carry direct implications for the calibration of credit risk models
in practice. Static-EAD models --- in which every loan is treated as a zero-coupon
bond with constant notional --- implicitly apply the Bullet schedule to all
instruments. When such models are used to compute economic capital or regulatory
stress tests for portfolios that are predominantly composed of amortising loans
(residential mortgages, consumer instalment credit, project finance), they
systematically \emph{overestimate} tail risk. Conversely, when the portfolio
contains instruments with payment-in-kind provisions, interest capitalisation, or
commitment facilities that can grow up to a reset date, a static model
\emph{underestimates} the tail by treating the current drawn balance as a fixed
exposure.

The simulation results quantify this misspecification: under a Balanced sector
allocation the static Bullet ES is $2.3$ times the Linear ES, implying that a
capital model that ignores amortisation may carry roughly twice as much economic
capital as is warranted for a fully amortising portfolio, or half as much as is
required for a negatively amortising one. The magnitude of this error is not
attenuated by sectoral diversification: the $2.3\times$ ratio is stable across
Balanced, Mixed, and Concentrated configurations, indicating that diversification
and amortisation are orthogonal risk-reduction mechanisms. Both should be accounted
for independently in any well-specified capital framework.

The finding that the Amplification Factor is governed by $c$ rather than by the EAD
profile suggests a clean separation of policy levers. Regulatory tools that operate
through collateral requirements, concentration limits, or macroprudential buffers
target $c$ and reduce the clustering multiplier; tools that mandate minimum
amortisation rates or restrict negatively amortising products target the EAD scale
and reduce the base level of all losses, including tail losses, without altering the
clustering structure.

\subsection{Directions for Future Work}

Several extensions of this framework merit investigation. The present analysis holds
the maturity $T$ fixed and equal across all schedules; allowing heterogeneous
maturities within a single portfolio would introduce maturity concentration as an
additional dimension of systemic risk, alongside sectoral concentration
\cite{jarratt2024}. Prepayment optionality --- which causes the effective EAD to
deviate from the contractual schedule when interest rates fall --- would introduce a
further source of path-dependence that interacts with the contagion feedback. Finally,
the theoretical ordering results of Section~\ref{sec:conclusion} rely on conditional
independence of default times given the sectoral paths; relaxing this assumption
through inter-sector contagion \cite{deshpande2009} or cross-obligor counterparty
channels would enrich the model but would require new comparison theorems adapted
to the resulting non-Markovian intensity dynamics.

% ============================================================
%  BIBLIOGRAPHY
% ============================================================

\begin{thebibliography}{99}

\bibitem{sinha2025}
D.~Sinha, S.~Sharma, S.~Jain, and S.~K.~Iyer,
``Simulation and Analysis of Sector-Apportioned Intensity Models for Correlated
Defaults,''
\textit{Working Paper}, 2025.

\bibitem{giesecke2011}
K.~Giesecke, B.~Kim, and S.~Zhu,
``Monte Carlo Algorithms for Default Timing Problems,''
\textit{Management Science}, vol.~57, no.~12, pp.~2115--2129, 2011.
\newblock \url{https://doi.org/10.1287/mnsc.1110.1411}

\bibitem{errais2010}
E.~Errais, K.~Giesecke, and L.~R.~Goldberg,
``Affine Point Processes and Portfolio Credit Risk,''
\textit{SIAM Journal on Financial Mathematics}, vol.~1, no.~1, pp.~642--665, 2010.
\newblock \url{https://doi.org/10.1137/090771272}

\bibitem{aitsahalia2015}
Y.~A\"{i}t-Sahalia and T.~R.~Hurd,
``Portfolio Choice in Markets with Contagion,''
\textit{Journal of Financial Econometrics}, vol.~14, no.~1, pp.~1--28, 2015.
\newblock \url{https://doi.org/10.1093/jjfinec/nbv024}

\bibitem{deshpande2009}
A.~Deshpande and S.~K.~Iyer,
``The Credit Risk$+$ Model with General Sector Correlations,''
\textit{Central European Journal of Operations Research}, vol.~17, no.~2,
pp.~219--228, 2009.
\newblock \url{https://doi.org/10.1007/s10100-009-0084-4}

\bibitem{vasicek2002}
O.~Vasicek,
``The Distribution of Loan Portfolio Value,''
\textit{Risk}, vol.~15, no.~12, pp.~160--162, 2002.

\bibitem{gupton1997}
G.~M.~Gupton, C.~C.~Finger, and M.~Bhatia,
``CreditMetrics Technical Document,''
J.P.\ Morgan \& Co., New York, 1997.
\newblock \url{https://www.msci.com/documents/10199/93396227-d449-4229-9143-24a94dab122f}

\bibitem{csfp1997}
Credit Suisse Financial Products,
\textit{CreditRisk$+$: A Credit Risk Management Framework},
Credit Suisse Financial Products, London, 1997.
\newblock \url{https://www.csfb.com/institutional/research/assets/creditrisk.pdf}

\bibitem{gordy2000}
M.~B.~Gordy,
``A Comparative Anatomy of Credit Risk Models,''
\textit{Journal of Banking \& Finance}, vol.~24, no.~1--2, pp.~119--149, 2000.
\newblock \url{https://doi.org/10.1016/S0378-4266(99)00054-0}

\bibitem{duffie1999}
D.~Duffie and K.~J.~Singleton,
``Modeling Term Structures of Defaultable Bonds,''
\textit{Review of Financial Studies}, vol.~12, no.~4, pp.~687--720, 1999.
\newblock \url{https://doi.org/10.1093/rfs/12.4.687}

\bibitem{lando2004}
D.~Lando,
\textit{Credit Risk Modeling: Theory and Applications},
Princeton University Press, Princeton, NJ, 2004.
\newblock ISBN: 978-0-691-08929-4

\bibitem{mcneil2015}
A.~J.~McNeil, R.~Frey, and P.~Embrechts,
\textit{Quantitative Risk Management: Concepts, Techniques and Tools},
revised ed., Princeton University Press, Princeton, NJ, 2015.
\newblock ISBN: 978-0-691-16627-8

\bibitem{bis2006}
Basel Committee on Banking Supervision,
``International Convergence of Capital Measurement and Capital Standards:
A Revised Framework (Basel~II),''
Bank for International Settlements, Basel, June 2006.
\newblock \url{https://www.bis.org/publ/bcbs128.htm}

\bibitem{gorton2010}
G.~B.~Gorton,
\textit{Slapped by the Invisible Hand: The Panic of 2007},
Oxford University Press, New York, 2010.
\newblock ISBN: 978-0-19-973415-7

\bibitem{brunnermeier2009}
M.~K.~Brunnermeier,
``Deciphering the Liquidity and Credit Crunch 2007--2008,''
\textit{Journal of Economic Perspectives}, vol.~23, no.~1, pp.~77--100, 2009.
\newblock \url{https://doi.org/10.1257/jep.23.1.77}

\bibitem{altman2005}
E.~I.~Altman, B.~Brady, A.~Resti, and A.~Sironi,
``The Link Between Default and Recovery Rates: Theory, Empirical Evidence,
and Implications,''
\textit{Journal of Business}, vol.~78, no.~6, pp.~2203--2227, 2005.
\newblock \url{https://doi.org/10.1086/497044}

\bibitem{frye2000}
J.~Frye,
``Depressing Recoveries,''
\textit{Risk}, vol.~13, no.~11, pp.~108--111, 2000.

\bibitem{schuermann2004}
T.~Schuermann,
``What Do We Know About Loss Given Default?''
in \textit{Credit Risk: Models and Management}, 2nd ed.,
D.~Shimko, Ed., Risk Books, London, 2004, pp.~249--274.
\newblock \url{https://doi.org/10.2139/ssrn.525262}

\bibitem{merton1974}
R.~C.~Merton,
``On the Pricing of Corporate Debt: The Risk Structure of Interest Rates,''
\textit{Journal of Finance}, vol.~29, no.~2, pp.~449--470, 1974.
\newblock \url{https://doi.org/10.1111/j.1540-6261.1974.tb03058.x}

\bibitem{jarratt2024}
J.~Jarratt,
``Sector Concentration Risk in Credit Portfolios,''
\textit{SSRN Working Paper}, 2024.
\newblock \url{https://doi.org/10.2139/ssrn.5064224}

\bibitem{shaked2007}
M.~Shaked and J.~G.~Shanthikumar,
\textit{Stochastic Orders},
Springer Series in Statistics, Springer, New York, 2007.
\newblock \url{https://doi.org/10.1007/978-0-387-34675-5}

\bibitem{follmer2016}
H.~F\"{o}llmer and A.~Schied,
\textit{Stochastic Finance: An Introduction in Discrete Time},
4th ed., De Gruyter, Berlin, 2016.
\newblock \url{https://doi.org/10.1515/9783110463453}

\bibitem{ikeda1989}
N.~Ikeda and S.~Watanabe,
\textit{Stochastic Differential Equations and Diffusion Processes},
2nd ed., North-Holland, Amsterdam, 1989.
\newblock ISBN: 978-0-444-87378-3

\end{thebibliography}

\end{document}
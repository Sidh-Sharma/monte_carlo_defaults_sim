\documentclass[preprint,11pt]{elsarticle}

\usepackage[letterpaper,margin=1in]{geometry}
\usepackage{microtype}
\usepackage{lmodern}
\newcommand{\para}[1]{\noindent\textbf{#1}\;}

\usepackage{amsmath,amssymb,amsthm,mathrsfs}

\theoremstyle{definition}
\newtheorem{definition}{Definition}[section]
\newtheorem{example}{Example}[section]

\theoremstyle{plain}
\newtheorem{theorem}{Theorem}[section]
\newtheorem{proposition}[theorem]{Proposition}

\theoremstyle{remark}
\newtheorem{remark}{Remark}[section]


\usepackage{titlesec}

\titlespacing*{\section}
{0pt}{2.0ex plus 0.8ex minus 0.3ex}{1.2ex}

\titlespacing*{\subsection}
{0pt}{1.5ex plus 0.6ex minus 0.2ex}{0.8ex}

\titlespacing*{\subsubsection}
{0pt}{1.2ex}{0.5ex}


\usepackage{parskip}
\setlength{\parskip}{0.6em}

\usepackage{booktabs,tabularx,multirow}
\setlength{\tabcolsep}{6pt}
\renewcommand{\arraystretch}{1.1}

\usepackage[font=small,labelfont=bf]{caption}

\usepackage[hidelinks]{hyperref}

\usepackage{fancyhdr}
\pagestyle{fancy}
\fancyhf{}
\fancyhead[R]{\thepage}

\usepackage[english]{babel}
\usepackage[autostyle,english=american]{csquotes}
\MakeOuterQuote{"}


\begin{document}

\begin{frontmatter}

\title{
Time-Varying Exposure and Systemic Risk:\\
Amortisation Schedules in Sector-Apportioned Intensity Models
}

\author{Devang Sinha}
\author{Siddharth Kamlesh Sharma}
\author{Shashi Jain}
\author{Srikanth K.\ Iyer}


\begin{abstract}
We extend the sector-apportioned intensity framework for correlated defaults
by introducing contractually specified, time-varying exposure at default (EAD)
and stochastic loss-given-default (LGD). Replacing static notional losses with
amortisation-dependent exposure schedules, we study Bullet, Linear (Italian),
French contracts within a contagion-sensitive
intensity model. Monte Carlo simulations show that repayment structure is the
dominant determinant of tail risk: Bullet portfolios exhibit Expected Shortfall
approximately 2.3 times larger than fully amortizing portfolios across all
sector concentration regimes. Amortisation also materially delays first passage
to systemic stress by attenuating contagion feedback, while default clustering
in the tail remains governed by the contagion parameter. The results have direct
implications for credit risk modelling and regulatory capital calibration.
\end{abstract}

\begin{keyword}
Credit risk \sep Default contagion \sep Amortisation \sep Expected shortfall \sep
Intensity models \sep Systemic risk
\end{keyword}

\end{frontmatter}
\section{Introduction}

\subsection{Motivation}

The sector-apportioned intensity framework developed in \cite{sinha2025} provides a tractable approach for analyzing correlated defaults arising from sectoral shocks. In that framework, each obligor's default intensity is decomposed as
\begin{equation}
\lambda^A_t \;=\; X^A \;+\; \sum_{j=1}^{J} w^{Aj}\,Y^j_t,
\label{eq:base_intensity}
\end{equation}
where $X^A$ represents a constant idiosyncratic component, $Y^j_t$ denotes the stochastic intensity of sector $j$ governed by a CIR-type diffusion with loss-driven feedback, and $w^{Aj} \geq 0$ measures the degree to which obligor $A$ is exposed to that sector. Portfolio losses enter the dynamics through a feedback term $\delta^{Aj}\,\mathrm{d}L_t$ in the evolution of each $Y^j_t$, generating the self-exciting default clustering that is characteristic of credit crises.

A central modeling simplification in \cite{sinha2025}—shared by much of the intensity-based credit risk literature—is that the loss incurred upon the $n$-th default is modeled as an independent draw $\ell_n \sim \mathcal{U}[0,1]$. This is equivalent to treating the exposure at default (EAD) of every obligor as constant through time and setting the loss given default (LGD) implicitly to unity. While this simplification does not distort the sector-concentration and contagion results that are the primary focus of that work, it abstracts away a crucial dimension of real-world credit risk: the outstanding principal of a loan evolves according to its contractual repayment schedule, and the profile of this evolution differs markedly across the loan structures that populate bank portfolios.

The present paper extends the framework of \cite{sinha2025} along two complementary dimensions. First, we replace the static, uniform loss draw with a \emph{time-varying} EAD schedule $E^A(t)$, so that the loss incurred when obligor $A$ defaults at time $\tau^A$ becomes
\begin{equation}
\ell^A \;=\; \mathrm{LGD}^A \cdot E^A(\tau^A),
\label{eq:loss_general}
\end{equation}
where $E^A(\tau^A)$ is the outstanding principal at the moment of default under the obligor's contracted repayment schedule. Second, we model LGD as stochastic, drawing $\mathrm{LGD}^A \sim \mathrm{Beta}(\alpha, \beta)$ independently for each default event, thereby replacing the implicit assumption of full loss on exposure.

The systematic intensity dynamics and contagion mechanism of \eqref{eq:base_intensity} are inherited unchanged; the contribution of this work lies entirely in enriching the loss side of the model. We study four canonical amortisation contracts—Bullet, Linear (Italian), French, and Negative Amortisation—and demonstrate that the choice of repayment structure has material consequences for portfolio tail risk, expected shortfall, and systemic resilience.

\subsection{Empirical and Regulatory Context}

The importance of time-varying exposure in credit portfolios has long been recognized in the regulatory capital literature. The Basel II and Basel III frameworks permit banks to use internal estimates of EAD for revolving facilities, acknowledging that the drawn amount at default need not equal the committed line \cite{bis2006}. For term loans and mortgages the outstanding balance follows a deterministic schedule conditional on no prepayment, yet the \emph{timing} of default relative to that schedule introduces material variability in realized losses. A borrower who defaults early in a bullet loan's life imposes a near-full-principal loss; the same event late in a well-amortized loan leaves far less outstanding. This simple observation implies that two portfolios with identical notional values and identical intensity processes can have substantially different loss distributions if their repayment architectures differ.

The interaction between amortization schedules and systemic risk is less well-studied, yet highly policy-relevant. During the 2007–2009 financial crisis, a disproportionate share of losses arose from instruments whose EAD was either constant to maturity or actively growing prior to a contractual reset \cite{brunnermeier2009}. By contrast, portfolios of traditional installment loans amortized steadily and presented declining exposures even as the underlying sectors came under stress. This empirical contrast motivates a structural comparison of repayment architectures within a contagion-sensitive intensity model.

Within the intensity-based literature, \cite{errais2010} model default clustering via self-exciting processes but maintain static notional exposures. The present work fills this gap by coupling the sector-apportioned intensity framework of \cite{sinha2025} with contractually specified, time-varying EAD schedules, thereby enabling a clean decomposition of how repayment structure interacts with contagion strength to shape the portfolio loss distribution.

A further departure from the base model concerns the recovery rate. Empirical evidence consistently documents that recovery rates are stochastic and, crucially, negatively correlated with default rates: recoveries fall precisely when defaults cluster, amplifying systemic losses \cite{altman2005}. To capture this effect in a tractable manner we model $\mathrm{LGD}^A \sim \mathrm{Beta}(\alpha, \beta)$, drawn independently across obligors and simulation paths. The Beta distribution is the canonical choice in both regulatory and academic practice \cite{bis2006}: it is supported on $[0,1]$, accommodates the full range of empirical skewness, and its parameters can be moment-matched directly from historical workout data. In accordance with standard regulatory assumptions we calibrate to a mean LGD of $45\%$ and a standard deviation of $15\%$, yielding $\alpha \approx 4.5$ and $\beta \approx 5.5$ via moment-matching.

% ============================================================
% SECTION 2: MODEL SETUP

\section{Model Setup and Paper Structure}

\subsection{Amortisation Schedules}

We consider a portfolio of $N$ obligors. Each obligor $A$ holds a loan of principal $P^A$, originated at $t=0$ with contractual maturity $T$. The outstanding balance (EAD) at time $t$ is denoted $E^A(t)$ and is determined entirely by the repayment contract. Schedules are implemented in discrete time with $N_p$ payment periods per year, so that the period length is $\Delta = 1/N_p$, the total number of periods is $n_{\mathrm{tot}} = \lceil T N_p \rceil$, and the number of payments made by time $t$ is $n(t) = \min\!\left(\lfloor t N_p \rfloor,\, n_{\mathrm{tot}}\right)$.

We study four canonical structures, each defined below for a loan with principal $P^A$ and contractual rate $r$.

\para{Bullet loan.} No principal is repaid prior to maturity; the full notional is outstanding until $T$:
\begin{equation}
E_{\mathrm{B}}(t) \;=\; \begin{cases} P^A, & t < T,\\ 0, & t \geq T. \end{cases}
\label{eq:bullet}
\end{equation}
The EAD profile is constant and the lender bears full principal exposure for the entire life of the loan.

\para{Linear (Italian) amortisation.} Principal is repaid in equal installments of size $P^A / n_{\mathrm{tot}}$ per period. The outstanding balance after $n(t)$ payments is
\begin{equation}
E_{\mathrm{L}}(t) \;=\; P^A\!\left(1 - \frac{n(t)}{n_{\mathrm{tot}}}\right).
\label{eq:linear}
\end{equation}
Exposure declines piecewise linearly, so that by mid-life roughly half the principal has been returned to the lender.

\para{French (constant-annuity) amortisation.} Total debt-service is constant across all periods. With per-period rate $r_\Delta = r / N_p$, the constant annuity payment is
\begin{equation}
C \;=\; P^A\,\frac{r_\Delta}{1 - (1+r_\Delta)^{-n_{\mathrm{tot}}}},
\label{eq:annuity_payment}
\end{equation}
and after $n(t)$ payments have been made, $n_r(t) = n_{\mathrm{tot}} - n(t)$ periods remain. The outstanding balance equals the present value of the remaining annuity stream:
\begin{equation}
E_{\mathrm{F}}(t) \;=\; C\,\frac{1 - (1+r_\Delta)^{-n_r(t)}}{r_\Delta}.
\label{eq:french}
\end{equation}
Early payments are predominantly interest, so principal declines slowly at first and accelerates toward maturity. This schedule governs the vast majority of residential mortgages and consumer installment loans.

\para{Negative amortisation.} Scheduled payments are insufficient to cover accruing interest; the shortfall is capitalized each period, so the outstanding balance grows monotonically throughout the loan's life:
\begin{equation}
E_{\mathrm{N}}(t) \;=\; \begin{cases} P^A\,(1 + r_\Delta)^{n(t)}, & t < T,\\ 0, & t \geq T. \end{cases}
\label{eq:negam}
\end{equation}
Any default occurring during $[0, T)$ is therefore associated with an exposure strictly exceeding the origination principal, making this the most credit-risk-adverse of the four structures. This contract appeared prominently in option-ARM mortgage products of the mid-2000s.

Figure~\ref{fig:ead_profiles} illustrates the four EAD profiles. The contrast between the declining profiles of the Linear and French schedules and the flat or growing profiles of the Bullet and Negative Amortisation contracts is the central object of study in this paper.

\begin{figure}[ht]
\centering
\includegraphics[width=0.7\textwidth]{figures/ead_profiles.png}
\caption{Outstanding balance profiles for four amortisation contracts over a five-year horizon. Parameters: $P = 10$, $T = 5$, $r = 0.12$, $N_p = 12$.}
\label{fig:ead_profiles}
\end{figure}

\subsection{Cumulative Exposure Mass}

Whilst the EAD profiles in Figure~\ref{fig:ead_profiles} characterize the \emph{instantaneous} outstanding balance under each contract, a complementary perspective is provided by the \emph{cumulative exposure mass}, defined as the time-integrated outstanding balance normalized by the product of principal and maturity:
\begin{equation}
\mathcal{M}(t) \;=\; \frac{1}{P\,T}\int_0^t E(s)\,\mathrm{d}s.
\label{eq:cumulative_mass}
\end{equation}
The quantity $\mathcal{M}(t)$ measures what fraction of the maximum possible credit exposure—a unit notional held for the full term—has been borne by the lender up to time $t$. Its derivative $\mathcal{M}'(t) = E(t)/(PT)$ is proportional to the instantaneous EAD, so the curvature of $\mathcal{M}(\cdot)$ directly reflects how rapidly outstanding principal is being returned.

Figure~\ref{fig:cumulative_mass} plots $\mathcal{M}(t)$ for the four contracts over the five-year horizon used in the experiments. Several features are noteworthy. The Bullet contract accumulates mass linearly, reflecting its constant exposure; its curve is a straight diagonal and serves as an upper reference. The Negative Amortisation contract exceeds even this benchmark for all $t \in (0,T)$, since capitalized interest causes the outstanding balance to grow above par throughout the loan's life. Both findings are consistent with the expectation that late- or non-amortizing structures impose the greatest credit exposure on the lender.

The Linear (Italian) and French curves lie below the Bullet, but differ subtly in their degree of concavity. The Linear schedule declines at a constant rate from origination, so $\mathcal{M}_L(t) = t/T - t^2/(2T^2)$ and the associated curvature $\mathcal{M}_L''(t) = -1/T^2$ is uniform throughout the horizon. The French schedule, by contrast, exhibits a convex EAD profile: early installments are predominantly interest, so $E_F(t)$ remains close to par in the first portion of the loan's life and only begins to fall steeply as the principal component of each payment grows. As a consequence, $\mathcal{M}_F(t)$ tracks the Bullet curve closely in early periods and curves away from it only in the later half of the horizon, concentrating its concavity toward maturity.

Integrated over the full term, the Linear contract therefore displays slightly more pronounced and uniformly distributed curvature than the French, even though the French contract ultimately bears a greater total exposure mass $\mathcal{M}_F(T) > \mathcal{M}_L(T)$. This distinction has practical implications for loss modeling: under the sector-apportioned intensity framework, defaults that arrive in the early-to-middle portion of the horizon—which are the most likely under typical CIR-level intensities—interact with an EAD that is, on average, higher for French than for Linear obligors, a difference that will be reflected in the tail metrics reported in Section~\ref{sec:tail}.

\begin{figure}[ht]
\centering
\includegraphics[width=0.7\textwidth]{figures/cum_exm_t.png}
\caption{Cumulative exposure mass $\mathcal{M}(t)$ over a five-year horizon. Linear exhibits more uniformly distributed concavity than French, whose curvature concentrates toward maturity. Negative Amortisation exceeds Bullet throughout due to interest capitalization.}
\label{fig:cumulative_mass}
\end{figure}

\subsection{Loss Structure and Modified Feedback}

The loss incurred when obligor $A$ defaults at time $\tau^A$ is given by
\begin{equation}
\ell^A \;=\; \mathrm{LGD}^A \cdot E^A(\tau^A),
\label{eq:lgd_full}
\end{equation}
where $\mathrm{LGD}^A \sim \mathrm{Beta}(4.5, 5.5)$ is drawn independently for each default event. The cumulative portfolio loss up to time $t$ retains the structure of \cite{sinha2025}:
\begin{equation}
L_t \;=\; \sum_{n \geq 1} \ell^{A_n}\,\mathbf{1}_{\{T_n \leq t\}},
\label{eq:loss_process}
\end{equation}
where $T_n$ is the $n$-th default time and $A_n$ is the identity of the defaulting obligor.

To ensure that the contagion signal remains comparable across repayment structures and maturities, the jump in the intensity of sector $j$ at default time $T_n$ is normalized by the origination principal of the defaulting obligor:
\begin{equation}
\Delta Y^j_{T_n} \;=\; \delta^{A_n j}\,\frac{\ell^{A_n}}{P^{A_n}}.
\label{eq:feedback_normalised}
\end{equation}
This preserves the structure of the original CIR-with-feedback dynamics while ensuring that $\delta^{A_n j}$ retains its interpretation as a dimensionless sensitivity parameter regardless of the repayment schedule.

\subsection{Parameterisation and Experimental Design}

The sectoral intensity dynamics follow the CIR-with-feedback specification of \cite{sinha2025} without modification. Portfolio size, sector-weight configurations, and idiosyncratic factor draws are held at the values reported in that paper: $N = 1000$ obligors, $J = 3$ sectors, $X^A \sim \mathcal{U}[0.01, 0.03]$. We consider three sector-weight configurations throughout the analysis. In the \emph{Balanced} configuration, $w^{Aj} = 1/3$ for all obligors $A$ and all sectors $j$, representing maximal diversification. In the \emph{Concentrated} configuration, $w^{A,1} = 0.7$ and $w^{A,2} = w^{A,3} = 0.15$, so that a dominant sector absorbs 70\% of the portfolio's exposure. The \emph{Mixed} configuration assigns $w^{A,1} = 0.5$, $w^{A,2} = 0.3$, and $w^{A,3} = 0.2$, representing partial concentration. All departures from the benchmark results of \cite{sinha2025} are therefore attributable solely to the introduction of time-varying EAD and stochastic LGD.

Amortisation parameters are set uniformly across the portfolio: principal $P = 10$, annual contractual rate $r = 0.12$, and $N_p = 12$ payment periods per year. Recovery parameters are calibrated as described above, yielding $(\alpha, \beta) \approx (4.5, 5.5)$. All results are based on 2{,}500 Monte Carlo replications using the asymptotically exact acceptance-rejection algorithm of \cite{giesecke2011}, adapted to evaluate time-dependent EAD at each accepted default time and to draw an independent Beta LGD for each default event.

\subsection{Paper Organisation}

The remainder of the paper is organized as follows. In Section~\ref{sec:tail} we quantify tail risk through the lens of Expected Shortfall (ES) and Value-at-Risk (VaR) across amortisation schedules and sectoral configurations at the terminal horizon $T=1$. We establish that repayment structure is the dominant determinant of tail severity, with Bullet ES approximately $2.3$ times Linear ES across all concentration regimes—a ratio that is remarkably stable and indicates that the protective effect of amortization is not contingent on sectoral diversification.

Section~\ref{sec:survival} shifts focus from terminal loss magnitudes to the temporal path by which portfolios arrive at stressed states. We analyze first-passage times to critical aggregate intensity thresholds over an extended horizon $T=2$, demonstrating that early principal repayment materially extends survival time and lowers threshold-crossing probability. The analysis reveals that amortization acts as a negative feedback brake on self-exciting intensity dynamics, delaying and in some cases entirely preventing the onset of systemic stress.

In Section~\ref{sec:contagion} we decompose the contagion mechanism through peak intensity and default clustering metrics. A key finding emerges: the Amplification Factor—measuring how much more defaults cluster in the tail relative to the unconditional mean—is approximately invariant to the EAD profile, governed instead by the contagion parameter $c$. This reveals that amortization does not alter the \emph{relative} clustering structure of the default process but rather acts as an \emph{exposure scale parameter}, attenuating every loss in the distribution proportionately without changing how defaults cluster.

Section~\ref{sec:conclusion} synthesizes these findings, discusses the theoretical foundation for the observed risk ordering through a convex ordering argument, and outlines implications for risk management and regulatory capital calibration. Additional experimental details, ablation studies focusing on alternative sector configurations and threshold specifications, and mathematical derivations appear in the appendices.

% ============================================================
% SECTION 3: TAIL RISK

\section{Tail Risk and Expected Shortfall}
\label{sec:tail}

\subsection{Mechanism}

To understand why repayment structure affects the tail of the loss distribution, it is useful to begin with the loss-generating mechanism directly. The portfolio loss at maturity is given by
\begin{equation}
L_T \;=\; \sum_{n=1}^{N(T)} \mathrm{LGD}^{A_n} \cdot E^{A_n}(T_n),
\label{eq:loss_mechanism}
\end{equation}
where $N(T)$ is the number of defaults in $[0,T]$ and $T_n$ is the time of the $n$-th default. Conditional on the intensity path $\{Y^j_t\}$, defaults cluster toward periods of elevated systemic intensity. Under the CIR-with-feedback dynamics of \eqref{eq:base_intensity}, each default triggers a jump $\Delta Y^j = \delta^{Aj} \ell^A / P^A$ in the sectoral intensities, making subsequent defaults more likely and generating the well-documented right-skew of credit loss distributions.

Crucially, the contribution of the $n$-th default to $L_T$ is $\mathrm{LGD}^{A_n} \cdot E^{A_n}(T_n)$: the systemic intensity and the EAD are evaluated at the \emph{same time} $T_n$. This temporal alignment is the key to understanding the differences across repayment structures.

For a Bullet obligor, $E_{\mathrm{B}}(T_n) = P^A$ regardless of when the default arrives, so the full principal is always at risk. For a negatively amortizing obligor, $E_{\mathrm{N}}(T_n) = P^A(1 + r_\Delta)^{n(T_n)} > P^A$, so late defaults—which are also the most likely under contagion-amplified intensity—carry an exposure that exceeds the original notional. In both cases, maximum loss coincides with the period of peak systemic stress.

By contrast, for Linear and French obligors the EAD is a declining function of $T_n$, so that a late-arriving, contagion-amplified default is partially offset by the exposure reduction that amortization has already delivered. The speed and convexity of this decline, as characterized by the cumulative exposure mass in Section~2.2, directly determines how much of this offset is realized.

\subsection{Risk Metrics under a Balanced Sector Allocation}

We begin by isolating the effect of repayment structure, fixing the sector-weight configuration to the Balanced case ($w^{Aj} = 1/J$ for all $A, j$) and the contagion parameter to $c = 0.01$. Table~\ref{tab:risk_metrics} reports the resulting risk metrics for a portfolio of $N = 1{,}000$ obligors at $T = 1$, under the stochastic Beta LGD framework with $(\alpha, \beta) \approx (4.5, 5.5)$.

\begin{table}[hbt!]
\centering
\caption{Portfolio risk metrics across amortisation schedules under a Balanced sector allocation.}
\label{tab:risk_metrics}
\begin{tabular}{lccccc}
\toprule
\textbf{Schedule} & \textbf{Mean Loss} & \textbf{Std Dev} & \textbf{VaR}$_{95\%}$ & \textbf{ES}$_{95\%}$ & \textbf{Excess ES} \\
\midrule
Bullet & 445.16 & 92.11 & 603.68 & 653.19 & 49.51 \\
French & 192.02 & 39.74 & 260.42 & 282.88 & 22.46 \\
Linear  & 187.92 & 39.41 & 254.87 & 277.08 & 22.21 \\
\bottomrule
\end{tabular}
\end{table}

The results establish a clear ordering. The Bullet schedule generates a mean loss of 445.16 with a 95\% Expected Shortfall of 653.19—more than double the ES of either amortizing schedule. The Linear contract attains the lowest ES at 277.08, with the French annuity structure following closely at 282.88. The moderate gap between the two amortizing schedules—approximately $2\%$ in ES terms—reflects the difference in their exposure profiles identified in Section~2.2: the French schedule maintains a higher outstanding balance in the early-to-middle portion of the horizon, where the probability of contagion-amplified defaults is concentrated, whereas the Linear schedule has already returned a proportionate share of principal by that point.

The \emph{Excess ES}, defined as $\text{ES}_{95\%} - \text{VaR}_{95\%}$, provides a measure of tail thickness conditional on a tail event having already occurred. Under the Bullet schedule, Excess ES reaches 49.51—more than double that of the amortizing schedules (22.46 and 22.21 respectively). This disproportionate expansion indicates that tail events under the Bullet are not merely more frequent but materially more severe: once the 95th percentile threshold is crossed, losses continue to accumulate substantially beyond it.

\subsection{Sectoral Configuration Impact}

The analysis in Table~\ref{tab:risk_metrics} holds the sectoral weight structure fixed. We now allow it to vary across the configurations introduced in Section~2.4: Balanced, Concentrated, and Mixed. Figures~\ref{fig:loss_concentrated} and~\ref{fig:loss_mixed} present the simulated loss distributions for the three completed schedules (Bullet, French, Linear) under Concentrated and Mixed sectoral configurations respectively.

Across both configurations, the qualitative ordering from Table~\ref{tab:risk_metrics} is preserved: the Bullet distribution consistently exhibits the heaviest right tail and the widest spread, with the two amortizing schedules forming a tightly clustered pair well below it. Within this stable ordering, sector concentration acts as an amplifier. Comparing the Concentrated configuration to the Balanced, the Bullet tail extends considerably further to the right, whilst the amortizing schedules, though also shifting outward, do so proportionately less.

This asymmetric amplification reflects the interaction between weight concentration and the contagion feedback of \eqref{eq:feedback_normalised}: a dominant sector absorbs a larger jump $\delta^{Aj} \ell^A / P^A$ at each default, which in turn elevates the intensities of the obligors most exposed to that sector—precisely those who carry the highest weight in a concentrated portfolio. Late defaults in a Bullet portfolio therefore arrive not only into an already-stressed intensity environment but with full exposure, compounding the tail loss beyond what would occur in a diversified setting.

\begin{figure}[hbt!]
\centering
\includegraphics[width=0.95\textwidth]{figures/loss_conc.png}
\caption{Loss distributions under Concentrated sector allocation ($w^{A,1} = 0.7$). Concentration amplifies the Bullet tail disproportionately; amortizing schedules remain relatively tight.}
\label{fig:loss_concentrated}
\end{figure}

\begin{figure}[hbt!]
\centering
\includegraphics[width=0.95\textwidth]{figures/loss_mixed.png}
\caption{Loss distributions under Mixed sector allocation ($w^{A,1} = 0.5$, $w^{A,2} = 0.3$, $w^{A,3} = 0.2$), displaying intermediate tail behavior.}
\label{fig:loss_mixed}
\end{figure}

The loss density plots also reveal a shift in distributional shape. Under the Balanced configuration (shown in Appendix~\ref{app:additional}), all three schedules produce approximately unimodal, right-skewed densities. Under the Concentrated configuration, the Bullet distribution develops a notably heavier and longer right tail, consistent with the heavy-tail amplification reported in \cite{sinha2025} for the concentrated sector case. The French and Linear distributions remain largely unimodal and relatively tight, confirming that the protective effect of continuous amortization is robust to sector concentration.

\subsection{Synthesis: ES across Schedules and Sectors}

Figure~\ref{fig:es_summary} consolidates the two dimensions of variation—repayment structure and sectoral allocation—into a single summary of ES$_{95\%}$ and Excess ES across the three primary sector configurations.

\begin{figure}[hbt!]
\centering
\includegraphics[width=0.75\textwidth]{figures/es-summary.png}
\caption{Expected Shortfall at 95\% (upper panel) and Excess ES (lower panel) by amortisation schedule across sector configurations. Repayment structure dominates ES level; concentration governs sensitivity.}
\label{fig:es_summary}
\end{figure}

Several conclusions emerge from this joint view. First, repayment structure is the dominant determinant of the level of ES. Across all sectoral configurations, the Bullet schedule consistently registers an ES approximately 2.3 times that of the Linear schedule and 2.3 times that of the French schedule, a ratio that is remarkably stable across sector configurations. This stability indicates that the protective effect of amortization is not contingent on any particular degree of sectoral diversification.

Second, sector concentration governs the \emph{sensitivity} of ES to the repayment schedule. In the Concentrated configuration, the absolute gap between Bullet ES and Linear ES is wider than in the Balanced case, confirming the amplification mechanism discussed above. The Mixed configuration registers intermediate gaps throughout, consistent with its partial concentration.

Third, the Excess ES panel reveals that the difference in tail \emph{thickness}—as opposed to tail \emph{level}—is also driven primarily by the repayment schedule. Even under the Balanced configuration, where sector diversification is maximal, the Bullet Excess ES is approximately double that of the amortizing schedules. This implies that the tail-thinning effect of amortization is structural and cannot be recovered by diversification alone.

% ============================================================
% SECTION 4: SURVIVAL PROBABILITY

\section{Survival Probability and Time to Systemic Stress}
\label{sec:survival}

\subsection{First Passage Time Framework}

Expected Shortfall characterizes the \emph{magnitude} of tail losses at the terminal horizon $T$, but it is silent on the \emph{temporal path} by which the portfolio arrives at a stressed state. From a risk management perspective, the speed with which a portfolio transitions into a systemic stress regime is at least as important as the eventual loss level: a portfolio that reaches critical intensity early in its life affords almost no time for supervisory intervention, whilst one that defers the onset of stress allows hedging and deleveraging to act.

The first passage time to a stress threshold is the natural object for capturing this dimension. We define the aggregate systemic intensity at time $t$ as
\begin{equation}
\mathcal{Y}_t \;=\; \sum_{j=1}^{J} Y^j_t,
\label{eq:aggregate_intensity}
\end{equation}
and the first passage time to a critical stress level $y^\star$ as
\begin{equation}
\tau^\star \;=\; \inf\!\left\{\, t > 0 \;\middle|\; \mathcal{Y}_t \geq y^\star \,\right\},
\label{eq:fpt}
\end{equation}
with $\tau^\star = +\infty$ on paths where $\mathcal{Y}_t < y^\star$ for all $t \in [0,T]$. The survival probability at time $t$ is then given by
\begin{equation}
S(t) \;=\; \mathbb{P}(\tau^\star > t).
\label{eq:survival}
\end{equation}

To ensure robustness of conclusions, we evaluate \eqref{eq:fpt} under a Deterministic threshold specification: $y^\star_{\mathrm{D}} = 0.75 \times \sum_{j=1}^{J} \lambda^j_{\max}$, a fixed proportion of the maximum theoretical baseline intensity. This threshold is independent of any particular Monte Carlo sample, making it transparent and replicable across parameterizations. Results under an alternative Dynamic Quantile threshold appear in Appendix~\ref{app:survival}.

For each schedule and sector configuration, we simulate $M = 2{,}500$ independent paths of the full system over an extended horizon $T=2$—double the terminal loss horizon used in Section~\ref{sec:tail}—to allow sufficient time for stress dynamics to develop. Each path yields a first passage time $\tau^\star_m \in (0, +\infty]$. The empirical survival function is estimated as
\begin{equation}
\hat{S}(t) \;=\; 1 \;-\; \frac{1}{M} \sum_{m=1}^{M} \mathbf{1}_{\{\tau^\star_m \leq t\}}.
\label{eq:empirical_survival}
\end{equation}
Two summary statistics are extracted for each configuration: the probability of first passage $\hat{p} = 1 - \hat{S}(T)$, measuring the fraction of paths that cross $y^\star$ within $[0, T]$; and the mean FPT conditional on crossing, $\hat{\mu} = \mathrm{mean}\{\tau^\star_m : \tau^\star_m < \infty\}$, which is the more informative summary when a substantial fraction of paths never cross the threshold.

\subsection{Mechanism: Amortisation Delays First Passage}

The connection between amortization and the first passage time runs directly through the normalized feedback of \eqref{eq:feedback_normalised}. Each default event at time $T_n$ generates a jump in the sectoral intensities of magnitude
\begin{equation}
\Delta Y^j_{T_n} \;=\; \delta^{A_n j}\,\frac{\mathrm{LGD}^{A_n} \cdot E^{A_n}(T_n)}{P^{A_n}}.
\label{eq:jump_fpt}
\end{equation}
For a Bullet obligor, $E^{A_n}(T_n) = P^{A_n}$ for all $T_n < T$, so the normalized jump is simply $\delta^{A_n j} \cdot \mathrm{LGD}^{A_n}$, independent of when the default occurs. For a Linear or French obligor, $E^{A_n}(T_n)/P^{A_n} < 1$ for any $T_n > 0$, and this ratio declines monotonically with $T_n$. It follows that, for any fixed realization of the LGD, an amortizing obligor generates a strictly smaller intensity jump than a Bullet obligor defaulting at the same time.

This attenuated jump slows the self-reinforcing feedback loop that drives $\mathcal{Y}_t$ towards $y^\star$, delaying and in some paths entirely preventing the threshold crossing. The attenuation is most pronounced for defaults that arrive late in the horizon, when contagion has already built up and the portfolio is closest to the threshold. For the Linear schedule, $E_{\mathrm{L}}(t)/P = 1 - t/T$, so a default at $t = 0.8T$ carries only 20\% of the full principal as contagion weight. For the French schedule, $E_{\mathrm{F}}(t)/P$ remains closer to unity throughout—reflecting the convex EAD profile—so the attenuation is less pronounced than for Linear in the early-to-middle part of the horizon but comparable near maturity. This difference mirrors the finding from Sections~2.2 and~\ref{sec:tail} that the Linear schedule provides marginally more tail protection than the French under the same contagion parameter.

\subsection{Results: Concentrated Sector Allocation}

Figure~\ref{fig:survival_con_d} presents the empirical survival curves under the Concentrated configuration ($w^{A,1} = 0.7$) with the Deterministic threshold.

\begin{figure}[hbt!]
\centering
\includegraphics[width=0.75\textwidth]{figures/survival_con_det.png}
\caption{Survival probability under Concentrated allocation over horizon $T=2$. Amortizing schedules maintain substantially higher terminal survival probability; gap widens under concentration.}
\label{fig:survival_con_d}
\end{figure}

Concentration has two compounding effects on the first passage time. First, it raises the baseline level of sectoral intensity: with the majority of obligors co-exposed to a single dominant sector, any default within that sector generates a large concentrated jump in $Y^1_t$, accelerating $\mathcal{Y}_t$ towards $y^\star$. Second, it reduces the stabilizing effect of amortization proportionately less than it accelerates the Bullet schedule. Because the dominant sector absorbs a fraction $w^{A,1} = 0.7$ of every contagion jump \eqref{eq:jump_fpt}, and because the Bullet schedule keeps this jump at its maximum level for all $T_n \in [0,T)$, the concentrated Bullet portfolio reaches the threshold with substantially higher probability and at substantially earlier times than in the Balanced case.

The amortizing schedules are also accelerated under concentration, but proportionately less so: the attenuation of their contagion jumps through declining EAD still operates, even if the dominant sector amplifies those jumps before attenuation. Consequently, under the Concentrated configuration the survival curves of all three schedules shift towards earlier first passage times and lower terminal survival probabilities $\hat{S}(T)$. The gap between the Bullet and the amortizing curves remains clearly visible and actually widens in absolute probability terms, confirming that the protective effect of amortization is not eroded by concentration but rather strengthened by it.

\subsection{Summary Metrics}

Table~\ref{tab:fpt_metrics} reports the two scalar FPT statistics for each combination of schedule and sector configuration under the Deterministic threshold.

\begin{table}[hbt!]
\centering
\caption{First passage time statistics across schedules and sector configurations.}
\label{tab:fpt_metrics}
\setlength{\tabcolsep}{10pt}
\begin{tabular}{llcc}
\toprule
\textbf{Sector} & \textbf{Schedule} & $\hat{p}$ & $\hat{\mu}$ \\
\midrule
\multirow{3}{*}{Balanced} & Bullet & 0.985 & 1.072 \\
& French & 0.174 & 1.057 \\
& Linear & 0.129 & 1.071 \\
\midrule
\multirow{3}{*}{Concentrated} & Bullet & 1.000 & 0.686 \\
& French & 0.854 & 0.878 \\
& Linear & 0.799 & 0.879 \\
\bottomrule
\end{tabular}
\end{table}

Several patterns deserve emphasis. The probability of first passage $\hat{p}$ is uniformly highest for the Bullet schedule and lowest for the Linear schedule across both configurations. Under the Balanced configuration the gap between Bullet and Linear first-passage probabilities is substantial (0.985 versus 0.129), confirming that a meaningful fraction of amortizing-portfolio paths avoid systemic stress entirely within the horizon. Under the Concentrated configuration, first-passage probabilities rise for all schedules, but the Bullet schedule consistently reaches values close to 1 whilst the amortizing schedules retain a lower probability (0.799 for Linear, 0.854 for French), reflecting the attenuated contagion mechanism.

The conditional mean FPT $\hat{\mu}$ is approximately equal across Linear and French in both cases, and strictly shorter for Bullet. This ordering is consistent with the EAD-attenuation mechanism: the Linear schedule reduces EAD at a constant rate from the outset, so its contagion-dampening effect is strongest in the early and middle portions of the horizon where crossing is most likely; the French schedule delivers comparable attenuation only later, when the annuity structure's accelerating amortization catches up. Under concentration, all conditional mean FPTs shorten, but the relative ordering and the magnitude of the Linear–Bullet gap are preserved.

% ============================================================
% SECTION 5: CONTAGION AMPLIFICATION

\section{Contagion Amplification and Default Clustering}
\label{sec:contagion}

\subsection{From Tail Losses to Intensity Dynamics}

Sections~\ref{sec:tail} and~\ref{sec:survival} established that repayment structure governs both the terminal loss distribution and the speed of transition to systemic stress. The present section examines the \emph{mechanism} by which these differences arise, decomposing the contagion channel into its intensity and clustering components.

The bridge between EAD and contagion is the normalized feedback of~\eqref{eq:feedback_normalised}. Each default at time $T_n$ inflicts a jump on the aggregate intensity $\mathcal{Y}_t = \sum_j Y^j_t$ of magnitude
\begin{equation}
\Delta\mathcal{Y}_{T_n} \;=\; \sum_{j=1}^{J} \delta^{A_n j}\, \frac{\mathrm{LGD}^{A_n} \cdot E^{A_n}(T_n)}{P^{A_n}}.
\label{eq:agg_jump}
\end{equation}
For a Bullet obligor, $E^{A_n}(T_n)/P^{A_n} = 1$ throughout, so every default delivers the maximum possible feedback regardless of timing. For a negatively amortizing obligor, this ratio exceeds unity, so defaults—which are themselves more likely when intensity is elevated—deliver \emph{super-par} jumps, creating a doubly reinforcing feedback loop. For Linear and French obligors the ratio is strictly less than one for any $T_n > 0$ and monotonically declining, attenuating each contagion jump relative to the Bullet benchmark.

This attenuation has a compounding character. Early defaults reduce $\mathcal{Y}_t$ below the Bullet trajectory, which in turn reduces the probability of subsequent defaults, which in turn reduces the future feedback into $\mathcal{Y}_t$. The amortization schedule therefore acts as a \emph{negative feedback brake} on the self-exciting intensity process, whilst the Bullet (and Negative Amortisation) schedules remove this brake entirely or reverse it. The consequence is not merely a shift in the loss distribution—it is a qualitative change in the dynamics of $\mathcal{Y}_t$ under stress, which we now quantify.

\subsection{Peak Systemic Intensity}

We characterize the severity of intensity excursions via the path-level maximum
\begin{equation}
\mathcal{Y}_{\max} \;=\; \sup_{t \in [0,T]} \mathcal{Y}_t,
\label{eq:ymax}
\end{equation}
which measures the most extreme stress state reached during each simulated scenario. Unlike the first passage time of Section~\ref{sec:survival}, which records \emph{when} a threshold is crossed, $\mathcal{Y}_{\max}$ records \emph{how far} the system departs from its baseline level over the full horizon.

Table~\ref{tab:clustering_metrics} reports the 99th percentile of $\mathcal{Y}_{\max}$ across 2{,}500 paths under the Balanced sector configuration. The Bullet schedule attains a 99th-percentile peak of $1.090$, more than double the corresponding figures for the Linear ($0.529$) and French ($0.560$) schedules. The ratio of $\approx 2\times$ in extreme peak intensity is consistent with the $\approx 2.3\times$ ratio in ES$_{95\%}$ reported in Table~\ref{tab:risk_metrics}: both reflect the same underlying difference in contagion jump magnitudes. The slight elevation of the French 99th-percentile relative to Linear ($0.560$ vs $0.529$) mirrors the finding from Sections~2.2 and~\ref{sec:tail} that the French schedule maintains higher outstanding balances in the early-to-middle portion of the horizon, where the intensity is most likely to be in a critical growth phase.

Figure~\ref{fig:ymax_scatter} presents a scatter plot of $\mathcal{Y}_{\max}$ against total portfolio loss $L_T$ across all simulated paths and schedules. The relationship is strongly positive in all cases, confirming that the loss level is a direct consequence of intensity excursion severity rather than an artifact of the LGD draw or the EAD profile in isolation.

\begin{figure}[hbt!]
\centering
\includegraphics[width=0.75\textwidth]{figures/ymax_scatter.png}
\caption{Peak intensity versus total loss under Balanced allocation. Amortizing schedules show compressed clouds with shallower slopes.}
\label{fig:ymax_scatter}
\end{figure}

Crucially, the clouds for the Bullet and Negative Amortisation schedules are shifted both rightward (higher $L_T$) and upward (higher $\mathcal{Y}_{\max}$) relative to the amortizing schedules, and the two dimensions of shift are proportionally aligned: larger intensity peaks produce larger total losses because the firms surviving to the peak are still exposed to near-full or super-par principal. Under the amortizing schedules the clouds are more compressed, and the slope of the $\mathcal{Y}_{\max}$–$L_T$ relationship is shallower: even when the intensity does spike, the reduced EAD limits the loss that the spike can generate. This compression is the geometric expression of the negative-feedback brake described above.

\subsection{Default Clustering and Amplification Factor}

Peak intensity characterizes aggregate system stress, but does not directly reveal how defaults \emph{cluster} within a scenario. To isolate the clustering effect we compare the unconditional expected number of defaults $\mathbb{E}[N(T)]$ with its conditional counterpart in the loss tail:
\begin{equation}
\mathbb{E}\!\left[\,N(T) \;\middle|\; L_T \geq \mathrm{VaR}_{95\%}\right],
\label{eq:conditional_defaults}
\end{equation}
where $\mathrm{VaR}_{95\%}$ is the schedule-specific 95th percentile loss. We define the \emph{Amplification Factor} as the ratio of these two expectations:
\begin{equation}
\mathcal{A} \;=\; \frac{\mathbb{E}[N(T) \mid L_T \geq \mathrm{VaR}_{95\%}]} {\mathbb{E}[N(T)]}.
\label{eq:amplification}
\end{equation}
A value $\mathcal{A} > 1$ indicates that tail loss scenarios are characterized by more defaults than a typical scenario—the signature of genuine default clustering rather than a single large loss event.

The results in Table~\ref{tab:clustering_metrics} reveal a striking pattern: the Amplification Factor is essentially stable across all three schedules, ranging narrowly from $1.328$ to $1.356$. This near-invariance is not coincidental. The Amplification Factor measures the \emph{relative} intensification of defaults in the tail and, under the affine jump-diffusion structure, this relative intensification is governed primarily by the contagion parameter $c$. Because $c = 0.01$ is held fixed across schedules, the multiplicative clustering behavior of the process is held fixed; only its \emph{scale} changes with the EAD profile.

\begin{table}[hbt!]
\centering
\caption{Default clustering and peak intensity metrics under Balanced allocation.}
\label{tab:clustering_metrics}
\setlength{\tabcolsep}{8pt}
\begin{tabular}{lcccc}
\toprule
\textbf{Schedule} & $\mathcal{Y}_{\max}^{(99)}$ & $\hat{\mathbb{E}}[N]$ & $\hat{\mathbb{E}}[N \mid \mathrm{Tail}]$ & $\hat{\mathcal{A}}$ \\
\midrule
Bullet & 1.090 & 265.47 & 358.69 & 1.351 \\
French & 0.560 & 191.15 & 259.19 & 1.356 \\
Linear & 0.529 & 186.63 & 247.82 & 1.328 \\
\bottomrule
\end{tabular}
\end{table}

The amortization schedule is therefore not a clustering parameter—it does not alter \emph{how much more} defaults cluster in the tail relative to the unconditional mean—but rather an \emph{exposure scale parameter} that determines the loss consequence of each default in the cluster.

This distinction has a precise implication. Because $\mathcal{A}$ is stable whilst $\mathbb{E}[N(T) \mid \text{Tail}]$ differs substantially across schedules—from 247.82 under Linear to 358.69 under Bullet, a difference of more than 110 defaults—the gap is entirely attributable to the difference in $\mathbb{E}[N(T)]$ itself. The Bullet portfolio simply experiences more defaults in expectation, because elevated EAD amplifies the contagion jump at each event, raising the subsequent intensity and accelerating future defaults. The higher conditional tail count is thus a consequence of the same positive feedback loop that elevates $\mathcal{Y}_{\max}$, not an independent clustering effect. Amortization suppresses the unconditional default rate, and this suppression scales proportionately into the tail.

% ============================================================
% SECTION 6: CONCLUSION

\section{Conclusion}
\label{sec:conclusion}

\subsection{Summary}

This paper has extended the sector-apportioned intensity framework of \cite{sinha2025} by replacing the static, uniform loss draw with contractually specified, time-varying EAD schedules and a Beta-distributed stochastic LGD. Three canonical amortisation structures—Bullet, Linear (Italian), French—were evaluated across sectoral concentration regimes using the asymptotically exact acceptance-rejection algorithm of \cite{giesecke2011}.

Three quantitative findings stand out. First, repayment structure is the dominant determinant of tail loss severity. The Bullet schedule generates a 95\% Expected Shortfall approximately $2.3$ times that of the Linear schedule under a Balanced sector allocation, a ratio that is remarkably stable across concentration regimes. This stability indicates that the protective effect of amortization operates independently of sectoral diversification.

Second, early principal repayment materially extends the system's survival time. Linear and French portfolios exhibit substantially lower first-passage probabilities and longer conditional mean passage times to the systemic stress threshold, regardless of which threshold specification is used. This temporal dimension is at least as important as the terminal loss magnitude from a risk management perspective, since it determines how much time is available for supervisory intervention before the portfolio enters a stressed regime.

Third, the Amplification Factor—the ratio of conditional-tail to unconditional expected defaults—is approximately invariant to the repayment schedule, ranging from $1.33$ to $1.36$ across all three completed schedules. This near-invariance reveals that amortization does not alter the clustering structure of the default process, which is governed by the contagion parameter $c$; it acts instead as an exposure scale parameter that attenuates every loss in the distribution, including those in the tail, proportionately and unconditionally.

\subsection{Risk Ordering}

Beyond the simulation evidence, the direction of the risk ordering across repayment structures can be established via a coupling argument rooted in the model's jump structure. Consider two portfolios, $\mathcal{P}_A$ and $\mathcal{P}_B$, driven by identical realizations of the Brownian motions $\{W^j_t\}$, idiosyncratic factors $\{X^A\}$, and LGD draws $\{\mathrm{LGD}^A\}$, but with exposure schedules satisfying $E^A_{\mathcal{P}_A}(t) \leq E^A_{\mathcal{P}_B}(t)$ for all obligors $A$ and all times $t \in [0,T]$.

Under this coupling, the first default time $T_1$ is identical across both portfolios, since the intensity process prior to the first event is driven entirely by the sectoral CIR dynamics and the idiosyncratic constants, which are common. However, the contagion jump at $T_1$ satisfies $\Delta\mathcal{Y}_{T_1}^{\mathcal{P}_B} \geq \Delta\mathcal{Y}_{T_1}^{\mathcal{P}_A}$ pathwise, because $E^{A_1}_{\mathcal{P}_B}(T_1) \geq E^{A_1}_{\mathcal{P}_A}(T_1)$ by assumption. After $T_1$, the sectoral intensity of $\mathcal{P}_B$ lies at or above that of $\mathcal{P}_A$ on every path. By the stochastic comparison theorem for jump-diffusions, this implies first-order stochastic dominance $L_T^{\mathcal{P}_B} \geq_{\mathrm{st}} L_T^{\mathcal{P}_A}$, which in turn gives convex ordering $L_T^{\mathcal{P}_A} \leq_{cx} L_T^{\mathcal{P}_B}$.

Since Expected Shortfall is a coherent risk measure that is monotone with respect to convex order, we obtain $\mathrm{ES}_\alpha(L_T^{\mathcal{P}_A}) \leq \mathrm{ES}_\alpha(L_T^{\mathcal{P}_B})$ for all $\alpha \in (0,1)$. The EAD pointwise ordering is satisfied by the schedule hierarchy identified empirically: $E_{\mathrm{Linear}}(t) \leq E_{\mathrm{French}}(t) \leq E_{\mathrm{Bullet}}(t) \leq E_{\mathrm{Neg.\,Am.}}(t)$ for all $t \in (0, T)$, yielding the ES ordering observed in the Monte Carlo results. The heavy tails are therefore not artifacts of the chosen parameterization; they are intrinsic mathematical properties of the contagion feedback loop when principal repayment is deferred. A detailed proof appears in ~\ref{app:math}.

\subsection{Implications for Risk Management}

These results carry direct implications for the calibration of credit risk models in practice. Static-EAD models—in which every loan is treated as a zero-coupon bond with constant notional—implicitly apply the Bullet schedule to all instruments. When such models are used to compute economic capital or regulatory stress tests for portfolios that are predominantly composed of amortizing loans (residential mortgages, consumer installment credit, project finance), they systematically \emph{overestimate} tail risk. Conversely, when the portfolio contains instruments with payment-in-kind provisions, interest capitalization, or commitment facilities that can grow up to a reset date, a static model \emph{underestimates} the tail by treating the current drawn balance as a fixed exposure.

The simulation results quantify this misspecification: under a Balanced sector allocation the static Bullet ES is $2.3$ times the Linear ES, implying that a capital model that ignores amortization may carry roughly twice as much economic capital as is warranted for a fully amortizing portfolio, or half as much as is required for a negatively amortizing one. The magnitude of this error is not attenuated by sectoral diversification: the $2.3\times$ ratio is stable across Balanced, Mixed, and Concentrated configurations, indicating that diversification and amortization are orthogonal risk-reduction mechanisms. Both should be accounted for independently in any well-specified capital framework.

The finding that the Amplification Factor is governed by $c$ rather than by the EAD profile suggests a clean separation of policy levers. Regulatory tools that operate through collateral requirements, concentration limits, or macroprudential buffers target $c$ and reduce the clustering multiplier; tools that mandate minimum amortization rates or restrict negatively amortizing products target the EAD scale and reduce the base level of all losses, including tail losses, without altering the clustering structure. The two mechanisms are complementary, not substitutable, and both should be actively managed in a comprehensive risk framework.

% ============================================================
% BIBLIOGRAPHY

\begin{thebibliography}{99}

\bibitem{sinha2025}
D.~Sinha, S.~Sharma, S.~Jain, and S.~K.~Iyer,
``Simulation and Analysis of Sector-Apportioned Intensity Models for Correlated Defaults,''
\textit{Working Paper}, 2025.

\bibitem{giesecke2011}
K.~Giesecke, B.~Kim, and S.~Zhu,
``Monte Carlo Algorithms for Default Timing Problems,''
\textit{Management Science}, vol.~57, no.~12, pp.~2115--2129, 2011.

\bibitem{errais2010}
E.~Errais, K.~Giesecke, and L.~R.~Goldberg,
``Affine Point Processes and Portfolio Credit Risk,''
\textit{SIAM Journal on Financial Mathematics}, vol.~1, no.~1, pp.~642--665, 2010.

\bibitem{bis2006}
Basel Committee on Banking Supervision,
``International Convergence of Capital Measurement and Capital Standards: A Revised Framework (Basel~II),''
Bank for International Settlements, Basel, June 2006.

\bibitem{brunnermeier2009}
M.~K.~Brunnermeier,
``Deciphering the Liquidity and Credit Crunch 2007--2008,''
\textit{Journal of Economic Perspectives}, vol.~23, no.~1, pp.~77--100, 2009.

\bibitem{altman2005}
E.~I.~Altman, B.~Brady, A.~Resti, and A.~Sironi,
``The Link Between Default and Recovery Rates: Theory, Empirical Evidence, and Implications,''
\textit{Journal of Business}, vol.~78, no.~6, pp.~2203--2227, 2005.

\end{thebibliography}
\newpage

% ============================================================
% APPENDICES

\appendix

\section{Additional Experimental Results}
\label{app:additional}

\subsection{Loss Distributions: Balanced and Random Configurations}

For completeness, we present loss distributions under Balanced and Random (Dirichlet-drawn) sector configurations. While these exhibit the same qualitative ordering as Concentrated and Mixed (Figures~\ref{fig:loss_concentrated}–\ref{fig:loss_mixed} in main text), the effects are less visually striking and therefore relegated here.

\begin{figure}[hbt!]
\centering
\includegraphics[width=0.9\textwidth]{figures/loss_balanced.png}
\caption{Loss distributions under Balanced sector allocation ($w^{Aj} = 1/3$, $c = 0.01$, $N = 1{,}000$, 2{,}500 paths). All three schedules produce approximately unimodal, right-skewed densities. Bullet exhibits wider spread and heavier right tail; French/Linear cluster tightly.}
\label{fig:loss_balanced_app}
\end{figure}

\begin{figure}[hbt!]
\centering
\includegraphics[width=0.9\textwidth]{figures/loss_random.png}
\caption{Loss distributions under Random (Dirichlet-drawn) sector allocation. Random configuration spans range of concentration levels; loss distribution lies between Balanced and Concentrated cases.}
\label{fig:loss_random_app}
\end{figure}

\subsection{Survival Probability: Dynamic Quantile Threshold}

Section~\ref{sec:survival} focused on Deterministic threshold results under Concentrated allocation as these displayed clearest separation between schedules. Here we present survival curves under Dynamic Quantile threshold ($y^\star_{\mathrm{Q}} = \hat{F}^{-1}_{Y_{\max}}(0.95)$, 95th percentile of baseline peak intensity distribution) for both Balanced and Concentrated configurations.

\para{Balanced Configuration.}
Figure~\ref{fig:survival_bal_q_app} shows survival under Balanced allocation, Dynamic Quantile threshold. Bullet curve declines from outset; amortising schedules display slower decline with non-trivial fraction surviving to maturity without crossing threshold (evidenced by $\hat{S}(T) > 0$ at right edge). However, Linear/French survival probabilities remain very high throughout (near 1.0 for most of horizon), making visual separation less striking than Concentrated case. This is why we prioritised Concentrated configuration in main text.

\begin{figure}[hbt!]
\centering
\includegraphics[width=0.75\textwidth]{figures/survival_bal_quantile.png}
\caption{Survival probability under Balanced allocation, Dynamic Quantile threshold ($y^\star_{\mathrm{Q}}$). Bullet crosses threshold markedly earlier than amortising schedules; fraction of Linear/French paths never cross within $[0,T]$. Note Linear/French survival probabilities remain near 1.0 throughout, making separation less visually dramatic than Concentrated case (Figure~\ref{fig:survival_con_d} main text).}
\label{fig:survival_bal_q_app}
\end{figure}

\para{Concentrated Configuration.}
Figure~\ref{fig:survival_con_q_app} shows Concentrated allocation under Dynamic Quantile threshold. Qualitatively similar to Deterministic threshold (Figure~\ref{fig:survival_con_d} main text) but with sample-dependent threshold calibration introducing additional variability in step functions.

\begin{figure}[hbt!]
\centering
\includegraphics[width=0.75\textwidth]{figures/survival_con_quantile.png}
\caption{Survival probability under Concentrated allocation, Dynamic Quantile threshold. Concentration accelerates first passage across all schedules; Bullet reaches $\hat{S}(t) \approx 0$ earlier than Balanced case. Amortising schedules retain higher terminal survival probability.}
\label{fig:survival_con_q_app}
\end{figure}

\subsection{Balanced Configuration Survival: Deterministic Threshold}

For completeness, survival curves under Balanced configuration with Deterministic threshold appear in Figure~\ref{fig:survival_bal_d_app}. Qualitative ordering preserved; smoother step functions owing to independence from sample variation.

\begin{figure}[hbt!]
\centering
\includegraphics[width=0.75\textwidth]{figures/survival_bal_det.png}
\caption{Survival probability under Balanced allocation, Deterministic threshold ($y^\star_{\mathrm{D}} = 0.75\, \sum_j \lambda^j_{\max}$). Ordering preserved; Deterministic threshold yields smoother steps. Linear registers slightly higher survival than French throughout horizon.}
\label{fig:survival_bal_d_app}
\end{figure}

\section{Survival Analysis: Extended Discussion}
\label{app:survival}

\subsection{Threshold Calibration Details}

Two threshold specifications probe different aspects of systemic stress:

\para{Dynamic Quantile Threshold.}
$y^\star_{\mathrm{Q}} = \hat{F}^{-1}_{Y_{\max}}\!(0.95)$, where $\hat{F}_{Y_{\max}}$ is empirical CDF of peak intensity under baseline (uniform LGD, constant EAD). This calibration ensures threshold represents genuinely extreme but historically grounded stress level rather than arbitrary scalar. Data-adaptive, tied to extreme realisations under \emph{base} model; crossing it under amortised schedule constitutes anomalous event.

\para{Deterministic Structural Threshold.}
$y^\star_{\mathrm{D}} = 0.75 \times \sum_{j=1}^{J} \lambda^j_{\max}$, where $\lambda^j_{\max}$ is upper level in acceptance-rejection algorithm for sector $j$. Independent of any Monte Carlo sample; transparent, replicable across parameterisations. Provides fixed structural benchmark for assessing amortisation's stabilising effect without dependence on baseline simulations.

Dynamic Quantile is data-adaptive; Deterministic is transparent and simulation-independent. Both yield qualitatively consistent conclusions (Tables~\ref{tab:fpt_metrics}, Figures~\ref{fig:survival_con_d}, \ref{fig:survival_bal_q_app}–\ref{fig:survival_bal_d_app}).

\section{Mathematical Derivations}
\label{app:math}

\subsection{Convex Ordering: Detailed Argument}

We provide full proof sketch of convex ordering result stated in Section~\ref{sec:conclusion}.

\begin{proposition}
Let $\mathcal{P}_A$, $\mathcal{P}_B$ be two portfolios with exposure schedules satisfying
\[
E^A_{\mathcal{P}_A}(t) \;\leq\; E^A_{\mathcal{P}_B}(t) \qquad \forall\, A,\, \forall\, t \in [0,T],
\]
and all other model primitives (Brownian motions, idiosyncratic factors, LGD draws) identical. Then
\[
L_T^{\mathcal{P}_A} \;\leq_{cx}\; L_T^{\mathcal{P}_B},
\]
implying $\mathrm{ES}_\alpha(L_T^{\mathcal{P}_A}) \leq \mathrm{ES}_\alpha(L_T^{\mathcal{P}_B})$ for all $\alpha \in (0,1)$.
\end{proposition}

\begin{proof}[Proof sketch]
\textbf{Step 1: Pathwise coupling.}
By construction, all randomness (Brownian increments $\{W^j_t\}$, $\{X^A\}$, $\{\mathrm{LGD}^A\}$) identical across $\mathcal{P}_A$, $\mathcal{P}_B$ on same probability space. Only difference is EAD schedule.

\textbf{Step 2: First default time.}
Intensity process prior to first default:
\[
\lambda^A_t \;=\; X^A \;+\; \sum_{j=1}^{J} w^{Aj}\,Y^j_t,
\]
with $Y^j_t$ following CIR dynamics driven by $W^j_t$. Since $\{W^j_t\}$, $\{X^A\}$ identical, intensity paths identical up to $T_1$. Thus $T_1$ same in both portfolios.

\textbf{Step 3: Contagion jump comparison.}
At $T_1$, defaulting obligor $A_1$ is the same (determined by intensity magnitudes, which are identical). 
The contagion jump in sector $j$ is
\[
\Delta Y^j_{T_1}(\mathcal{P}_B)
\;=\;
\delta^{A_1 j}\,\frac{\mathrm{LGD}^{A_1} \cdot E^{A_1}_{\mathcal{P}_B}(T_1)}{P^{A_1}}
\;\geq\;
\delta^{A_1 j}\,\frac{\mathrm{LGD}^{A_1} \cdot E^{A_1}_{\mathcal{P}_A}(T_1)}{P^{A_1}}
\;=\;
\Delta Y^j_{T_1}(\mathcal{P}_A),
\]
because $E^{A_1}_{\mathcal{P}_B}(T_1) \geq E^{A_1}_{\mathcal{P}_A}(T_1)$ by assumption and $\mathrm{LGD}^{A_1}$ is identical.

\textbf{Step 4: Intensity dominance propagation.}
Post-$T_1$, for $t > T_1$:
\[
Y^j_t(\mathcal{P})
\;=\;
Y^j_{T_1^-}
\;+\;
\Delta Y^j_{T_1}(\mathcal{P})
\;+\;
\int_{T_1}^t (\cdots)\,\mathrm{d}s
\;+\;
\int_{T_1}^t \sigma^j\sqrt{Y^j_s(\mathcal{P})}\,\mathrm{d}W^j_s.
\]
Since the Brownian increment $\int_{T_1}^t \mathrm{d}W^j_s$ is the same, the drift has the same functional form, and
$Y^j_{T_1^-}$ is identical, but
\[
\Delta Y^j_{T_1}(\mathcal{P}_B) \;\geq\; \Delta Y^j_{T_1}(\mathcal{P}_A),
\]
the comparison theorem for SDEs yields
\[
Y^j_t(\mathcal{P}_B) \;\geq\; Y^j_t(\mathcal{P}_A)
\quad \text{pathwise for all } t > T_1.
\]

\textbf{Step 5: Subsequent defaults.}
Higher intensity $\lambda^A_t(\mathcal{P}_B) \geq \lambda^A_t(\mathcal{P}_A)$ implies defaults arrive earlier in $\mathcal{P}_B$ (in stochastic sense). Each subsequent default in $\mathcal{P}_B$ inflicts larger contagion jump than corresponding default in $\mathcal{P}_A$ (by EAD ordering), further widening intensity gap. By induction, intensity dominance propagates through all default events.

\textbf{Step 6: Loss process dominance.}
Cumulative loss:
\[
L_T \;=\; \sum_{n=1}^{N(T)} \mathrm{LGD}^{A_n} \cdot E^{A_n}(T_n).
\]
Since $N(T)$ is stochastically larger in $\mathcal{P}_B$ (higher intensity), and each individual loss $\mathrm{LGD}^{A_n} \cdot E^{A_n}(T_n)$ weakly larger in $\mathcal{P}_B$ (EAD ordering), we have
\[
L_T^{\mathcal{P}_B} \;\geq_{\mathrm{st}}\; L_T^{\mathcal{P}_A}.
\]

\textbf{Step 7: Convex ordering.}
First-order stochastic dominance $\geq_{\mathrm{st}}$ implies convex order $\leq_{cx}$ (standard result in stochastic orders literature). Since ES is coherent risk measure monotone with respect to convex order:
\[
\mathrm{ES}_\alpha(L_T^{\mathcal{P}_A}) \;\leq\; \mathrm{ES}_\alpha(L_T^{\mathcal{P}_B}) \quad \forall\, \alpha \in (0,1).
\]
\end{proof}

\para{Application to schedule hierarchy.}
The EAD pointwise ordering
\[
E_{\mathrm{Linear}}(t) \;\leq\; E_{\mathrm{French}}(t) \;\leq\; E_{\mathrm{Bullet}}(t) \;\leq\; E_{\mathrm{Neg.\,Am.}}(t) \qquad \forall\, t \in (0, T)
\]
directly satisfies proposition hypothesis, yielding ES ordering:
\[
\mathrm{ES}_\alpha(L_T^{\mathrm{Linear}}) \;\leq\; \mathrm{ES}_\alpha(L_T^{\mathrm{French}}) \;\leq\; \mathrm{ES}_\alpha(L_T^{\mathrm{Bullet}}) \;\leq\; \mathrm{ES}_\alpha(L_T^{\mathrm{Neg.\,Am.}}).
\]

\para{Caveats.}
(i) Coupling argument relies on pathwise intensity ordering holding between successive defaults, guaranteed under CIR by comparison theorem but may require separate verification for more general diffusions. (ii) Inequality in EAD hierarchy strict for all $t \in (0, T)$ but collapses at $t = T$ for Bullet ($E_{\mathrm{B}}(T) = 0$); ordering is property of horizon interior, not endpoint.

\section{Parameterisation Details}
\label{app:params}

\subsection{Sector Intensity Dynamics}

Each sector $j \in \{1, 2, 3\}$ governed by CIR process with loss-driven jump:
\[
\mathrm{d}Y^j_t \;=\; \kappa^j(\theta^j - Y^j_t)\,\mathrm{d}t \;+\; \sigma^j\sqrt{Y^j_t}\,\mathrm{d}W^j_t \;+\; \sum_{n : A_n \in \text{sector } j} \delta^{A_n j}\,\frac{\ell^{A_n}}{P^{A_n}}\,\mathrm{d}N_t,
\]
where $\kappa^j = 0.5$ (mean reversion speed), $\theta^j = 0.05$ (long-run mean), $\sigma^j = 0.15$ (volatility), $Y^j_0 = 0.03$ (initial level).

Contagion parameter $\delta^{Aj} = c \cdot w^{Aj}$ with $c = 0.01$ (base contagion strength) and $w^{Aj}$ the sector weight of obligor $A$ in sector $j$.

\subsection{Sector Weight Configurations}

\para{Balanced.} $w^{Aj} = 1/3$ for all obligors $A$, all sectors $j$. Maximal diversification.

\para{Concentrated.} $w^{A,1} = 0.7$, $w^{A,2} = 0.15$, $w^{A,3} = 0.15$ for all $A$. Dominant sector absorbs 70\% exposure.

\para{Mixed.} $w^{A,1} = 0.5$, $w^{A,2} = 0.3$, $w^{A,3} = 0.2$ for all $A$. Partial concentration.

\para{Random.} Sector weights $(w^{A,1}, w^{A,2}, w^{A,3})$ drawn independently per obligor from $\mathrm{Dirichlet}(1, 1, 1)$ (uniform over simplex).

\subsection{Simulation Algorithm}

Monte Carlo simulation via Giesecke-Kim-Zhu acceptance-rejection algorithm \cite{giesecke2011}:
\begin{enumerate}
\item Set upper bounds $\lambda^j_{\max}$ on sectoral intensities.
\item Simulate candidate default times from Poisson process with intensity $\sum_j \lambda^j_{\max}$.
\item At each candidate time, accept/reject based on ratio of true intensity to upper bound.
\item Upon acceptance, sample defaulting obligor proportionally to individual intensities, draw $\mathrm{LGD}^A \sim \mathrm{Beta}(4.5, 5.5)$, compute loss $\ell^A = \mathrm{LGD}^A \cdot E^A(t)$, update sectoral intensities via jump $\Delta Y^j = \delta^{Aj} \ell^A / P^A$.
\item Continue until horizon $T$ reached.
\end{enumerate}

Asymptotically exact as $\lambda^j_{\max} \to \infty$; in practice, $\lambda^j_{\max}$ chosen conservatively to ensure acceptance rate $> 90\%$.

\end{document}